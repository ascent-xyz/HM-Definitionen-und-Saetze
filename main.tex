\documentclass[11pt,a4paper]{report}

\usepackage[tmargin=2cm,rmargin=1in,lmargin=1in,margin=0.85in,bmargin=2cm,footskip=.2in]{geometry}

\usepackage{amsmath,amsthm,mathtools}
\usepackage{mleftright}
\usepackage{fixdif,derivative}
\usepackage{physics2}
\usephysicsmodule{ab.legacy}
\usepackage{twemojis}

% SCHRIFTEN - MUSS NACH MATHTOOLS GELADEN WERDEN, SONST IST Z. B. \UNDEBRACE FEHLERHAFT!!!
\PassOptionsToPackage{math-style=ISO}{unicode-math}
\usepackage[default]{fontsetup}

%\setmainfont{NewCM10-Book.otf} %Hier keine Mathematikschrift benutzen!
%\setmathfont{NewCMMath-Book}
%\setmathfont{NewCMMath-Book}[range=cal, StylisticSet=1]
%\setmathfont{NewCMMath-Book}[range=scr]


\usepackage[ngerman]{babel}
\usepackage{csquotes}
\usepackage{titlesec}
\titleformat{\chapter}[hang]
  {\normalfont\LARGE\bfseries}
  {}
  {0pt}
  {}

\usepackage{microtype}
\usepackage[no-script, no-inner, no-close]{innerscript}

%\usepackage[varbb]{newpxmath}
\usepackage{xfrac}
\usepackage[makeroom]{cancel}
%\usepackage{mathtools}
\usepackage{bookmark}
\usepackage{enumitem}
\usepackage{hyperref,theoremref}
\hypersetup{
	pdftitle={Definitionen-und-Saetze-der-HM1},
	colorlinks=true, linkcolor=doc!90,
	bookmarksnumbered=true,
	bookmarksopen=true
}
\usepackage[most,many,breakable]{tcolorbox}
\usepackage{xcolor}
\usepackage{varwidth}
\usepackage{varwidth}
\usepackage{etoolbox}
%\usepackage{authblk}
\usepackage{nameref}
\usepackage{multicol,array}
\usepackage[ruled,vlined,linesnumbered]{algorithm2e}
\usepackage{comment} % enables the use of multi-line comments (\ifx \fi)
\usepackage{import}
\usepackage{xifthen}
\usepackage{pdfpages}
\usepackage{transparent}

\newcommand\mycommfont[1]{\footnotesize\ttfamily\textcolor{blue}{#1}}
\SetCommentSty{mycommfont}
\newcommand{\incfig}[1]{%
    \def\svgwidth{\columnwidth}
    \import{./figures/}{#1.pdf_tex}
}

\usepackage{tikzsymbols}
\renewcommand\qedsymbol{$\Laughey$}


%\usepackage{import}
%\usepackage{xifthen}
%\usepackage{pdfpages}
%\usepackage{transparent}


%%%%%%%%%%%%%%%%%%%%%%%%%%%%%%
% SELF MADE COLORS
%%%%%%%%%%%%%%%%%%%%%%%%%%%%%%



\definecolor{myg}{RGB}{56, 140, 70}
\definecolor{myb}{RGB}{45, 111, 177}
\definecolor{myr}{RGB}{199, 68, 64}
\definecolor{mytheorembg}{HTML}{F2F2F9}
\definecolor{mytheoremfr}{HTML}{00007B}
\definecolor{mylenmabg}{HTML}{FFFAF8}
\definecolor{mylenmafr}{HTML}{983b0f}
\definecolor{mypropbg}{HTML}{f2fbfc}
\definecolor{mypropfr}{HTML}{191971}
\definecolor{myexamplebg}{HTML}{F2FBF8}
\definecolor{myexamplefr}{HTML}{88D6D1}
\definecolor{myexampleti}{HTML}{2A7F7F}
\definecolor{mydefinitbg}{HTML}{E5E5FF}
\definecolor{mydefinitfr}{HTML}{3F3FA3}
\definecolor{notesgreen}{RGB}{0,162,0}
\definecolor{myp}{RGB}{197, 92, 212}
\definecolor{mygr}{HTML}{2C3338}
\definecolor{myred}{RGB}{127,0,0}
\definecolor{myyellow}{RGB}{169,121,69}
\definecolor{myexercisebg}{HTML}{F2FBF8}
\definecolor{myexercisefg}{HTML}{88D6D1}



%%%%%%%%%%%%%%%%%%%%%%%%%%%%
% TCOLORBOX SETUPS
%%%%%%%%%%%%%%%%%%%%%%%%%%%%

\setlength{\parindent}{1cm}
%================================
% THEOREM BOX
%================================

\tcbuselibrary{theorems,skins,hooks}
\newtcbtheorem[number within=section]{Satz}{Satz}
{%
	enhanced,
	breakable,
	colback = mytheorembg,
	frame hidden,
	boxrule = 0sp,
	borderline west = {2pt}{0pt}{mytheoremfr},
	sharp corners,
	detach title,
	before upper = \tcbtitle\par\smallskip,
	coltitle = mytheoremfr,
	fonttitle = \bfseries\sffamily,
	description font = \mdseries,
	separator sign none,
	segmentation style={solid, mytheoremfr},
}
{th}

\tcbuselibrary{theorems,skins,hooks}
\newtcbtheorem[number within=chapter]{theorem}{Theorem}
{%
	enhanced,
	breakable,
	colback = mytheorembg,
	frame hidden,
	boxrule = 0sp,
	borderline west = {2pt}{0pt}{mytheoremfr},
	sharp corners,
	detach title,
	before upper = \tcbtitle\par\smallskip,
	coltitle = mytheoremfr,
	fonttitle = \bfseries\sffamily,
	description font = \mdseries,
	separator sign none,
	segmentation style={solid, mytheoremfr},
}
{th}


\tcbuselibrary{theorems,skins,hooks}
\newtcolorbox{Theoremcon}
{%
	enhanced
	,breakable
	,colback = mytheorembg
	,frame hidden
	,boxrule = 0sp
	,borderline west = {2pt}{0pt}{mytheoremfr}
	,sharp corners
	,description font = \mdseries
	,separator sign none
}

%================================
% Corollery
%================================
\tcbuselibrary{theorems,skins,hooks}
\newtcbtheorem[number within=section]{Corollary}{Corollary}
{%
	enhanced
	,breakable
	,colback = myp!10
	,frame hidden
	,boxrule = 0sp
	,borderline west = {2pt}{0pt}{myp!85!black}
	,sharp corners
	,detach title
	,before upper = \tcbtitle\par\smallskip
	,coltitle = myp!85!black
	,fonttitle = \bfseries\sffamily
	,description font = \mdseries
	,separator sign none
	,segmentation style={solid, myp!85!black}
}
{th}
\tcbuselibrary{theorems,skins,hooks}
\newtcbtheorem[number within=chapter]{corollary}{Corollary}
{%
	enhanced
	,breakable
	,colback = myp!10
	,frame hidden
	,boxrule = 0sp
	,borderline west = {2pt}{0pt}{myp!85!black}
	,sharp corners
	,detach title
	,before upper = \tcbtitle\par\smallskip
	,coltitle = myp!85!black
	,fonttitle = \bfseries\sffamily
	,description font = \mdseries
	,separator sign none
	,segmentation style={solid, myp!85!black}
}
{th}


%================================
% LEMMA
%================================

\tcbuselibrary{theorems,skins,hooks}
\newtcbtheorem[number within=section]{Lemma}{Lemma}
{%
	enhanced,
	breakable,
	colback = mylenmabg,
	frame hidden,
	boxrule = 0sp,
	borderline west = {2pt}{0pt}{mylenmafr},
	sharp corners,
	detach title,
	before upper = \tcbtitle\par\smallskip,
	coltitle = mylenmafr,
	fonttitle = \bfseries\sffamily,
	description font = \mdseries,
	separator sign none,
	segmentation style={solid, mylenmafr},
}
{th}

\tcbuselibrary{theorems,skins,hooks}
\newtcbtheorem[number within=chapter]{lenma}{Lenma}
{%
	enhanced,
	breakable,
	colback = mylenmabg,
	frame hidden,
	boxrule = 0sp,
	borderline west = {2pt}{0pt}{mylenmafr},
	sharp corners,
	detach title,
	before upper = \tcbtitle\par\smallskip,
	coltitle = mylenmafr,
	fonttitle = \bfseries\sffamily,
	description font = \mdseries,
	separator sign none,
	segmentation style={solid, mylenmafr},
}
{th}

%================================
% Exercise
%================================

\tcbuselibrary{theorems,skins,hooks}
\newtcbtheorem[number within=section]{Exercise}{Exercise}
{%
	enhanced,
	breakable,
	colback = myexercisebg,
	frame hidden,
	boxrule = 0sp,
	borderline west = {2pt}{0pt}{myexercisefg},
	sharp corners,
	detach title,
	before upper = \tcbtitle\par\smallskip,
	coltitle = myexercisefg,
	fonttitle = \bfseries\sffamily,
	description font = \mdseries,
	separator sign none,
	segmentation style={solid, myexercisefg},
}
{th}

\tcbuselibrary{theorems,skins,hooks}
\newtcbtheorem[number within=chapter]{exercise}{Exercise}
{%
	enhanced,
	breakable,
	colback = myexercisebg,
	frame hidden,
	boxrule = 0sp,
	borderline west = {2pt}{0pt}{myexercisefg},
	sharp corners,
	detach title,
	before upper = \tcbtitle\par\smallskip,
	coltitle = myexercisefg,
	fonttitle = \bfseries\sffamily,
	description font = \mdseries,
	separator sign none,
	segmentation style={solid, myexercisefg},
}
{th}



%================================
% PROPOSITION
%================================

\tcbuselibrary{theorems,skins,hooks}
\newtcbtheorem[number within=section]{Prop}{Proposition}
{%
	enhanced,
	breakable,
	colback = mypropbg,
	frame hidden,
	boxrule = 0sp,
	borderline west = {2pt}{0pt}{mypropfr},
	sharp corners,
	detach title,
	before upper = \tcbtitle\par\smallskip,
	coltitle = mypropfr,
	fonttitle = \bfseries\sffamily,
	description font = \mdseries,
	separator sign none,
	segmentation style={solid, mypropfr},
}
{th}

\tcbuselibrary{theorems,skins,hooks}
\newtcbtheorem[number within=chapter]{prop}{Proposition}
{%
	enhanced,
	breakable,
	colback = mypropbg,
	frame hidden,
	boxrule = 0sp,
	borderline west = {2pt}{0pt}{mypropfr},
	sharp corners,
	detach title,
	before upper = \tcbtitle\par\smallskip,
	coltitle = mypropfr,
	fonttitle = \bfseries\sffamily,
	description font = \mdseries,
	separator sign none,
	segmentation style={solid, mypropfr},
}
{th}


%================================
% CLAIM
%================================

\tcbuselibrary{theorems,skins,hooks}
\newtcbtheorem[number within=section]{claim}{Claim}
{%
	enhanced
	,breakable
	,colback = myg!10
	,frame hidden
	,boxrule = 0sp
	,borderline west = {2pt}{0pt}{myg}
	,sharp corners
	,detach title
	,before upper = \tcbtitle\par\smallskip
	,coltitle = myg!85!black
	,fonttitle = \bfseries\sffamily
	,description font = \mdseries
	,separator sign none
	,segmentation style={solid, myg!85!black}
}
{th}




%================================
% EXAMPLE BOX
%================================

\newtcbtheorem[number within=section]{Example}{Example}
{%
	colback = myexamplebg
	,breakable
	,colframe = myexamplefr
	,coltitle = myexampleti
	,boxrule = 1pt
	,sharp corners
	,detach title
	,before upper=\tcbtitle\par\smallskip
	,fonttitle = \bfseries
	,description font = \mdseries
	,separator sign none
	,description delimiters parenthesis
}
{ex}

\newtcbtheorem[number within=chapter]{example}{Example}
{%
	colback = myexamplebg
	,breakable
	,colframe = myexamplefr
	,coltitle = myexampleti
	,boxrule = 1pt
	,sharp corners
	,detach title
	,before upper=\tcbtitle\par\smallskip
	,fonttitle = \bfseries
	,description font = \mdseries
	,separator sign none
	,description delimiters parenthesis
}
{ex}

%================================
%  BOX
%================================

\newtcbtheorem[number within=section]{Definition}{Definition}{enhanced,
	before skip=2mm,after skip=2mm, colback=red!5,colframe=red!80!black,boxrule=0.5mm,
	attach boxed title to top left={xshift=1cm,yshift*=1mm-\tcboxedtitleheight}, varwidth boxed title*=-3cm,
	boxed title style={frame code={
					\path[fill=tcbcolback]
					([yshift=-1mm,xshift=-1mm]frame.north west)
					arc[start angle=0,end angle=180,radius=1mm]
					([yshift=-1mm,xshift=1mm]frame.north east)
					arc[start angle=180,end angle=0,radius=1mm];
					\path[left color=tcbcolback!60!black,right color=tcbcolback!60!black,
						middle color=tcbcolback!80!black]
					([xshift=-2mm]frame.north west) -- ([xshift=2mm]frame.north east)
					[rounded corners=1mm]-- ([xshift=1mm,yshift=-1mm]frame.north east)
					-- (frame.south east) -- (frame.south west)
					-- ([xshift=-1mm,yshift=-1mm]frame.north west)
					[sharp corners]-- cycle;
				},interior engine=empty,
		},
	fonttitle=\bfseries,
	title={#2},#1}{def}
\newtcbtheorem[number within=chapter]{definition}{Definition}{enhanced,
	before skip=2mm,after skip=2mm, colback=red!5,colframe=red!80!black,boxrule=0.5mm,
	attach boxed title to top left={xshift=1cm,yshift*=1mm-\tcboxedtitleheight}, varwidth boxed title*=-3cm,
	boxed title style={frame code={
					\path[fill=tcbcolback]
					([yshift=-1mm,xshift=-1mm]frame.north west)
					arc[start angle=0,end angle=180,radius=1mm]
					([yshift=-1mm,xshift=1mm]frame.north east)
					arc[start angle=180,end angle=0,radius=1mm];
					\path[left color=tcbcolback!60!black,right color=tcbcolback!60!black,
						middle color=tcbcolback!80!black]
					([xshift=-2mm]frame.north west) -- ([xshift=2mm]frame.north east)
					[rounded corners=1mm]-- ([xshift=1mm,yshift=-1mm]frame.north east)
					-- (frame.south east) -- (frame.south west)
					-- ([xshift=-1mm,yshift=-1mm]frame.north west)
					[sharp corners]-- cycle;
				},interior engine=empty,
		},
	fonttitle=\bfseries,
	title={#2},#1}{def}



%================================
% EXERCISE BOX
%================================

\makeatletter
\newtcbtheorem{question}{Question}{enhanced,
	breakable,
	colback=white,
	colframe=myb!80!black,
	attach boxed title to top left={yshift*=-\tcboxedtitleheight},
	fonttitle=\bfseries,
	title={#2},
	boxed title size=title,
	boxed title style={%
			sharp corners,
			rounded corners=northwest,
			colback=tcbcolframe,
			boxrule=0pt,
		},
	underlay boxed title={%
			\path[fill=tcbcolframe] (title.south west)--(title.south east)
			to[out=0, in=180] ([xshift=5mm]title.east)--
			(title.center-|frame.east)
			[rounded corners=\kvtcb@arc] |-
			(frame.north) -| cycle;
		},
	#1
}{def}
\makeatother

%================================
% SOLUTION BOX
%================================

\makeatletter
\newtcolorbox{solution}{enhanced,
	breakable,
	colback=white,
	colframe=myg!80!black,
	attach boxed title to top left={yshift*=-\tcboxedtitleheight},
	title=Solution,
	boxed title size=title,
	boxed title style={%
			sharp corners,
			rounded corners=northwest,
			colback=tcbcolframe,
			boxrule=0pt,
		},
	underlay boxed title={%
			\path[fill=tcbcolframe] (title.south west)--(title.south east)
			to[out=0, in=180] ([xshift=5mm]title.east)--
			(title.center-|frame.east)
			[rounded corners=\kvtcb@arc] |-
			(frame.north) -| cycle;
		},
}
\makeatother

%================================
% Question BOX
%================================

\makeatletter
\newtcbtheorem{qstion}{Question}{enhanced,
	breakable,
	colback=white,
	colframe=mygr,
	attach boxed title to top left={yshift*=-\tcboxedtitleheight},
	fonttitle=\bfseries,
	title={#2},
	boxed title size=title,
	boxed title style={%
			sharp corners,
			rounded corners=northwest,
			colback=tcbcolframe,
			boxrule=0pt,
		},
	underlay boxed title={%
			\path[fill=tcbcolframe] (title.south west)--(title.south east)
			to[out=0, in=180] ([xshift=5mm]title.east)--
			(title.center-|frame.east)
			[rounded corners=\kvtcb@arc] |-
			(frame.north) -| cycle;
		},
	#1
}{def}
\makeatother

\newtcbtheorem[number within=chapter]{wconc}{Wrong Concept}{
	breakable,
	enhanced,
	colback=white,
	colframe=myr,
	arc=0pt,
	outer arc=0pt,
	fonttitle=\bfseries\sffamily\large,
	colbacktitle=myr,
	attach boxed title to top left={},
	boxed title style={
			enhanced,
			skin=enhancedfirst jigsaw,
			arc=3pt,
			bottom=0pt,
			interior style={fill=myr}
		},
	#1
}{def}



%================================
% NOTE BOX
%================================

\usetikzlibrary{arrows,calc,shadows.blur}
\tcbuselibrary{skins}
\newtcolorbox{note}[1][]{%
	enhanced jigsaw,
	colback=gray!20!white,%
	colframe=gray!80!black,
	size=small,
	boxrule=1pt,
	title=\textbf{Note:-},
	halign title=flush center,
	coltitle=black,
	breakable,
	drop shadow=black!50!white,
	attach boxed title to top left={xshift=1cm,yshift=-\tcboxedtitleheight/2,yshifttext=-\tcboxedtitleheight/2},
	minipage boxed title=1.5cm,
	boxed title style={%
			colback=white,
			size=fbox,
			boxrule=1pt,
			boxsep=2pt,
			underlay={%
					\coordinate (dotA) at ($(interior.west) + (-0.5pt,0)$);
					\coordinate (dotB) at ($(interior.east) + (0.5pt,0)$);
					\begin{scope}
						\clip (interior.north west) rectangle ([xshift=3ex]interior.east);
						\filldraw [white, blur shadow={shadow opacity=60, shadow yshift=-.75ex}, rounded corners=2pt] (interior.north west) rectangle (interior.south east);
					\end{scope}
					\begin{scope}[gray!80!black]
						\fill (dotA) circle (2pt);
						\fill (dotB) circle (2pt);
					\end{scope}
				},
		},
	#1,
}

%%%%%%%%%%%%%%%%%%%%%%%%%%%%%%
% SELF MADE COMMANDS
%%%%%%%%%%%%%%%%%%%%%%%%%%%%%%


\newcommand{\thm}[2]{\begin{Satz}{#1}{}#2\end{Satz}}
\newcommand{\cor}[2]{\begin{Corollary}{#1}{}#2\end{Corollary}}
\newcommand{\mlenma}[2]{\begin{Lemma}{#1}{}#2\end{Lemma}}
\newcommand{\mer}[2]{\begin{Exercise}{#1}{}#2\end{Exercise}}
\newcommand{\mprop}[2]{\begin{Prop}{#1}{}#2\end{Prop}}
\newcommand{\clm}[3]{\begin{claim}{#1}{#2}#3\end{claim}}
\newcommand{\wc}[2]{\begin{wconc}{#1}{}\setlength{\parindent}{1cm}#2\end{wconc}}
\newcommand{\thmcon}[1]{\begin{Theoremcon}{#1}\end{Theoremcon}}
\newcommand{\ex}[2]{\begin{Example}{#1}{}#2\end{Example}}
\newcommand{\dfn}[2]{\begin{Definition}[colbacktitle=red!75!black]{#1}{}#2\end{Definition}}
\newcommand{\dfnc}[2]{\begin{definition}[colbacktitle=red!75!black]{#1}{}#2\end{definition}}
\newcommand{\qs}[2]{\begin{question}{#1}{}#2\end{question}}
\newcommand{\pf}[2]{\begin{myproof}[#1]#2\end{myproof}}
\newcommand{\nt}[1]{\begin{note}#1\end{note}}

\newcommand*\circled[1]{\tikz[baseline=(char.base)]{
		\node[shape=circle,draw,inner sep=1pt] (char) {#1};}}
\newcommand\getcurrentref[1]{%
	\ifnumequal{\value{#1}}{0}
	{??}
	{\the\value{#1}}%
}
\newcommand{\getCurrentSectionNumber}{\getcurrentref{section}}
\newenvironment{myproof}[1][\proofname]{%
	\proof[\bfseries #1: ]%
}{\endproof}

\newcommand{\mclm}[2]{\begin{myclaim}[#1]#2\end{myclaim}}
\newenvironment{myclaim}[1][\claimname]{\proof[\bfseries #1: ]}{}
\newenvironment{iclaim}[1][\claimname]{\bfseries #1\mdseries:}{}
\newcommand{\iclm}[2]{\begin{iclaim}[#1]#2\end{iclaim}}

\newcounter{mylabelcounter}

\makeatletter
\newcommand{\setword}[2]{%
	\phantomsection
	#1\def\@currentlabel{\unexpanded{#1}}\label{#2}%
}
\makeatother




\tikzset{
	symbol/.style={
			draw=none,
			every to/.append style={
					edge node={node [sloped, allow upside down, auto=false]{$#1$}}}
		}
}


% deliminators
%\DeclarePairedDelimiter{\abs}{\lvert}{\rvert}
%\DeclarePairedDelimiter{\norm}{\lVert}{\rVert}

%\DeclarePairedDelimiter{\ceil}{\lceil}{\rceil}
%\DeclarePairedDelimiter{\floor}{\lfloor}{\rfloor}
%\DeclarePairedDelimiter{\round}{\lfloor}{\rceil}

\newsavebox\diffdbox
\newcommand{\slantedromand}{{\mathpalette\makesl{d}}}
\newcommand{\makesl}[2]{%
\begingroup
\sbox{\diffdbox}{$\mathsurround=0pt#1\mathrm{#2}$}%
\pdfsave
\pdfsetmatrix{1 0 0.2 1}%
\rlap{\usebox{\diffdbox}}%
\pdfrestore
\hskip\wd\diffdbox
\endgroup
}
\newcommand{\dd}[1][]{\ensuremath{\mathop{}\!\ifstrempty{#1}{%
\slantedromand\@ifnextchar^{\hspace{0.2ex}}{\hspace{0.1ex}}}%
{\slantedromand\hspace{0.2ex}^{#1}}}}
\ProvideDocumentCommand\dv{o m g}{%
  \ensuremath{%
    \IfValueTF{#3}{%
      \IfNoValueTF{#1}{%
        \frac{\dd #2}{\dd #3}%
      }{%
        \frac{\dd^{#1} #2}{\dd #3^{#1}}%
      }%
    }{%
      \IfNoValueTF{#1}{%
        \frac{\dd}{\dd #2}%
      }{%
        \frac{\dd^{#1}}{\dd #2^{#1}}%
      }%
    }%
  }%
}
\providecommand*{\pdv}[3][]{\frac{\partial^{#1}#2}{\partial#3^{#1}}}
%  - others
\DeclareMathOperator{\Lap}{\mathcal{L}}
\DeclareMathOperator{\Var}{Var} % varience
\DeclareMathOperator{\Cov}{Cov} % covarience
\DeclareMathOperator{\E}{E} % expected

% Since the amsthm package isn't loaded

% I prefer the slanted \leq
\let\oldleq\leq % save them in case they're every wanted
\let\oldgeq\geq
\renewcommand{\leq}{\leqslant}
\renewcommand{\geq}{\geqslant}

% % redefine matrix env to allow for alignment, use r as default
% \renewcommand*\env@matrix[1][r]{\hskip -\arraycolsep
%     \let\@ifnextchar\new@ifnextchar
%     \array{*\c@MaxMatrixCols #1}}


%\usepackage{framed}
%\usepackage{titletoc}
%\usepackage{etoolbox}
%\usepackage{lmodern}


%\patchcmd{\tableofcontents}{\contentsname}{\sffamily\contentsname}{}{}

%\renewenvironment{leftbar}
%{\def\FrameCommand{\hspace{6em}%
%		{\color{myyellow}\vrule width 2pt depth 6pt}\hspace{1em}}%
%	\MakeFramed{\parshape 1 0cm \dimexpr\textwidth-6em\relax\FrameRestore}\vskip2pt%
%}
%{\endMakeFramed}

%\titlecontents{chapter}
%[0em]{\vspace*{2\baselineskip}}
%{\parbox{4.5em}{%
%		\hfill\Huge\sffamily\bfseries\color{myred}\thecontentspage}%
%	\vspace*{-2.3\baselineskip}\leftbar\textsc{\small\chaptername~\thecontentslabel}\\\sffamily}
%{}{\endleftbar}
%\titlecontents{section}
%[8.4em]
%{\sffamily\contentslabel{3em}}{}{}
%{\hspace{0.5em}\nobreak\itshape\color{myred}\contentspage}
%\titlecontents{subsection}
%[8.4em]
%{\sffamily\contentslabel{3em}}{}{}
%{\hspace{0.5em}\nobreak\itshape\color{myred}\contentspage}



%%%%%%%%%%%%%%%%%%%%%%%%%%%%%%%%%%%%%%%%%%%
% TABLE OF CONTENTS
%%%%%%%%%%%%%%%%%%%%%%%%%%%%%%%%%%%%%%%%%%%

\usepackage{tikz}
\definecolor{doc}{RGB}{0,60,110}
\usepackage{titletoc}
\contentsmargin{0cm}
\titlecontents{chapter}[3.7pc]
{\addvspace{30pt}%
	\begin{tikzpicture}[remember picture, overlay]%
		\draw[fill=doc!60,draw=doc!60] (-7,-.1) rectangle (-0.9,.5);%
		\pgftext[left,x=-3.7cm,y=0.2cm]{\color{white}\Large\sc\bfseries Kapitel\ \thecontentslabel};%
	\end{tikzpicture}\color{doc!60}\large\sc\bfseries}%
{}
{}
{\;\titlerule\;\large\sc\bfseries Seite \thecontentspage
	\begin{tikzpicture}[remember picture, overlay]
		\draw[fill=doc!60,draw=doc!60] (2pt,0) rectangle (4,0.1pt);
	\end{tikzpicture}}%
\titlecontents{section}[3.7pc]
{\addvspace{2pt}}
{\contentslabel[\thecontentslabel]{2pc}}
{}
{\hfill\small \thecontentspage}
[]
\titlecontents*{subsection}[3.7pc]
{\addvspace{-1pt}\small}
{}
{}
{\ --- \small\thecontentspage}
[ \textbullet\ ][]

\makeatletter
\renewcommand{\tableofcontents}{%
	\chapter*{%
	  \vspace*{-20\p@}%
	  \begin{tikzpicture}[remember picture, overlay]%
		  \pgftext[right,x=15cm,y=0.2cm]{\color{doc!60}\Huge\sc\bfseries \contentsname};%
		  \draw[fill=doc!60,draw=doc!60] (13,-.75) rectangle (20,1);%
		  \clip (13,-.75) rectangle (20,1);
		  \pgftext[right,x=15cm,y=0.2cm]{\color{white}\Huge\sc\bfseries \contentsname};%
	  \end{tikzpicture}}%
	\@starttoc{toc}}
\makeatother

\usepackage{scalerel}
%%%%%%%%%%%%%%%%%%%%%%%%%%%%%%
% GENERAL

\newcommand{\e}{\mathrm{e}} % Der hier von \eul redefined
\newcommand{\imag}{\mathrm{i}}
%\newcommand{\dif}[1]{\mathrm{d}#1} not needed (fixdif package)
\newcommand{\Le}{\mleft}
\newcommand{\Ri}{\mright}
\newcommand{\RR}[1][]{\ensuremath{\ifstrempty{#1}{\mathbb{R}}{\mathbb{R}^{#1}}}}
\newcommand{\ensp}{\enspace}
\newcommand{\sbst}{\subset}
\newcommand{\id}{\mathbb{1}}
\renewcommand{\implies}{\Rightarrow}
\newcommand{\openint}[2]{{]{#1,#2}[}}
\newcommand{\diag}[1]{\mathrm{diag}\Le\{#1\Ri\}}
\DeclarePairedDelimiterXPP\Exp[1]{\operatorname{exp}}{(}{)}{}{#1} %exp Operator mit passendem Klammern Spacing oder so
\newcommand{\bv}[1]{\symbfit{#1}} % bolt vector

%%%%%%%%%%%%%%%%%%%%%%%%%%%%%%
% SYMBOLS

\newcommand{\veps}{\varepsilon}
\newcommand{\vphi}{\varphi}

%%%%%%%%%%%%%%%%%%%%%%%%%%%%%%
% TOPOLGY

\DeclareMathOperator{\accumulated}{acc}
\DeclareMathOperator{\isolated}{iso}
\DeclareMathOperator{\interior}{int}
\DeclareMathOperator{\exterior}{ext}
\newcommand{\edge}{\partial}
\DeclareMathOperator{\CauchySeq}{\text{CF}}

%%%%%%%%%%%%%%%%%%%%%%%%%%%%%%
% LINEAR ALGEBRA

\DeclareMathOperator{\trace}{Tr}
\DeclareMathOperator*{\Vektor}{\scalerel*{V}{\sum}}
\DeclareMathOperator{\core}{ker}
%\DeclareMathOperator{\span}{span}

%%%%%%%%%%%%%%%%%%%%%%%%%%%%%%
% VECTOR CALCULUS

\newcommand{\nbl}{\symbfup{\nabla}}
\DeclareMathOperator{\grad}{grad}
\DeclareMathOperator{\divgc}{div}
\DeclareMathOperator{\rot}{rot}

%%%%%%%%%%%%%%%%%%%%%%%%%%%%%%
% PHYSICS
\newcommand{\poiss}[2]{\Le\{#1,#2\Ri\}} % Poissant brackets

%\prime
%\dprime
%\trprime
% number sets
%\newcommand{\RR}[1][]{\ensuremath{\ifstrempty{#1}{\mathbb{R}}{\mathbb{R}^{#1}}}}
\newcommand{\NN}[1][]{\ensuremath{\ifstrempty{#1}{\mathbb{N}}{\mathbb{N}^{#1}}}}
\newcommand{\CC}[1][]{\ensuremath{\ifstrempty{#1}{\mathbb{C}}{\mathbb{C}^{#1}}}}
\newcommand{\KK}[1][]{\ensuremath{\ifstrempty{#1}{\mathbb{K}}{\mathbb{K}^{#1}}}}
\newcommand{\ZZ}[1][]{\ensuremath{\ifstrempty{#1}{\mathbb{Z}}{\mathbb{Z}^{#1}}}}
\newcommand{\QQ}[1][]{\ensuremath{\ifstrempty{#1}{\mathbb{Q}}{\mathbb{Q}^{#1}}}}
\newcommand{\PP}[1][]{\ensuremath{\ifstrempty{#1}{\mathbb{P}}{\mathbb{P}^{#1}}}}
\newcommand{\HH}[1][]{\ensuremath{\ifstrempty{#1}{\mathbb{H}}{\mathbb{H}^{#1}}}}
\newcommand{\FF}[1][]{\ensuremath{\ifstrempty{#1}{\mathbb{F}}{\mathbb{F}^{#1}}}}
% expected value
\newcommand{\EE}{\ensuremath{\mathbb{E}}}

%---------------------------------------
% BlackBoard Math Fonts :-
%---------------------------------------

%Captital Letters
\newcommand{\bbA}{\mathbb{A}}	\newcommand{\bbB}{\mathbb{B}}
\newcommand{\bbC}{\mathbb{C}}	\newcommand{\bbD}{\mathbb{D}}
\newcommand{\bbE}{\mathbb{E}}	\newcommand{\bbF}{\mathbb{F}}
\newcommand{\bbG}{\mathbb{G}}	\newcommand{\bbH}{\mathbb{H}}
\newcommand{\bbI}{\mathbb{I}}	\newcommand{\bbJ}{\mathbb{J}}
\newcommand{\bbK}{\mathbb{K}}	\newcommand{\bbL}{\mathbb{L}}
\newcommand{\bbM}{\mathbb{M}}	\newcommand{\bbN}{\mathbb{N}}
\newcommand{\bbO}{\mathbb{O}}	\newcommand{\bbP}{\mathbb{P}}
\newcommand{\bbQ}{\mathbb{Q}}	\newcommand{\bbR}{\mathbb{R}}
\newcommand{\bbS}{\mathbb{S}}	\newcommand{\bbT}{\mathbb{T}}
\newcommand{\bbU}{\mathbb{U}}	\newcommand{\bbV}{\mathbb{V}}
\newcommand{\bbW}{\mathbb{W}}	\newcommand{\bbX}{\mathbb{X}}
\newcommand{\bbY}{\mathbb{Y}}	\newcommand{\bbZ}{\mathbb{Z}}

%---------------------------------------
% MathCal Fonts :-
%---------------------------------------

%Captital Letters
\newcommand{\mcA}{\mathcal{A}}	\newcommand{\mcB}{\mathcal{B}}
\newcommand{\mcC}{\mathcal{C}}	\newcommand{\mcD}{\mathcal{D}}
\newcommand{\mcE}{\mathcal{E}}	\newcommand{\mcF}{\mathcal{F}}
\newcommand{\mcG}{\mathcal{G}}	\newcommand{\mcH}{\mathcal{H}}
\newcommand{\mcI}{\mathcal{I}}	\newcommand{\mcJ}{\mathcal{J}}
\newcommand{\mcK}{\mathcal{K}}	\newcommand{\mcL}{\mathcal{L}}
\newcommand{\mcM}{\mathcal{M}}	\newcommand{\mcN}{\mathcal{N}}
\newcommand{\mcO}{\mathcal{O}}	\newcommand{\mcP}{\mathcal{P}}
\newcommand{\mcQ}{\mathcal{Q}}	\newcommand{\mcR}{\mathcal{R}}
\newcommand{\mcS}{\mathcal{S}}	\newcommand{\mcT}{\mathcal{T}}
\newcommand{\mcU}{\mathcal{U}}	\newcommand{\mcV}{\mathcal{V}}
\newcommand{\mcW}{\mathcal{W}}	\newcommand{\mcX}{\mathcal{X}}
\newcommand{\mcY}{\mathcal{Y}}	\newcommand{\mcZ}{\mathcal{Z}}



%---------------------------------------
% Bold Math Fonts :-
%---------------------------------------

%Captital Letters
\newcommand{\bmA}{\boldsymbol{A}}	\newcommand{\bmB}{\boldsymbol{B}}
\newcommand{\bmC}{\boldsymbol{C}}	\newcommand{\bmD}{\boldsymbol{D}}
\newcommand{\bmE}{\boldsymbol{E}}	\newcommand{\bmF}{\boldsymbol{F}}
\newcommand{\bmG}{\boldsymbol{G}}	\newcommand{\bmH}{\boldsymbol{H}}
\newcommand{\bmI}{\boldsymbol{I}}	\newcommand{\bmJ}{\boldsymbol{J}}
\newcommand{\bmK}{\boldsymbol{K}}	\newcommand{\bmL}{\boldsymbol{L}}
\newcommand{\bmM}{\boldsymbol{M}}	\newcommand{\bmN}{\boldsymbol{N}}
\newcommand{\bmO}{\boldsymbol{O}}	\newcommand{\bmP}{\boldsymbol{P}}
\newcommand{\bmQ}{\boldsymbol{Q}}	\newcommand{\bmR}{\boldsymbol{R}}
\newcommand{\bmS}{\boldsymbol{S}}	\newcommand{\bmT}{\boldsymbol{T}}
\newcommand{\bmU}{\boldsymbol{U}}	\newcommand{\bmV}{\boldsymbol{V}}
\newcommand{\bmW}{\boldsymbol{W}}	\newcommand{\bmX}{\boldsymbol{X}}
\newcommand{\bmY}{\boldsymbol{Y}}	\newcommand{\bmZ}{\boldsymbol{Z}}
%Small Letters
\newcommand{\bma}{\boldsymbol{a}}	\newcommand{\bmb}{\boldsymbol{b}}
\newcommand{\bmc}{\boldsymbol{c}}	\newcommand{\bmd}{\boldsymbol{d}}
\newcommand{\bme}{\boldsymbol{e}}	\newcommand{\bmf}{\boldsymbol{f}}
\newcommand{\bmg}{\boldsymbol{g}}	\newcommand{\bmh}{\boldsymbol{h}}
\newcommand{\bmi}{\boldsymbol{i}}	\newcommand{\bmj}{\boldsymbol{j}}
\newcommand{\bmk}{\boldsymbol{k}}	\newcommand{\bml}{\boldsymbol{l}}
\newcommand{\bmm}{\boldsymbol{m}}	\newcommand{\bmn}{\boldsymbol{n}}
\newcommand{\bmo}{\boldsymbol{o}}	\newcommand{\bmp}{\boldsymbol{p}}
\newcommand{\bmq}{\boldsymbol{q}}	\newcommand{\bmr}{\boldsymbol{r}}
\newcommand{\bms}{\boldsymbol{s}}	\newcommand{\bmt}{\boldsymbol{t}}
\newcommand{\bmu}{\boldsymbol{u}}	\newcommand{\bmv}{\boldsymbol{v}}
\newcommand{\bmw}{\boldsymbol{w}}	\newcommand{\bmx}{\boldsymbol{x}}
\newcommand{\bmy}{\boldsymbol{y}}	\newcommand{\bmz}{\boldsymbol{z}}

%---------------------------------------
% Scr Math Fonts :-
%---------------------------------------

\newcommand{\sA}{{\mathscr{A}}}   \newcommand{\sB}{{\mathscr{B}}}
\newcommand{\sC}{{\mathscr{C}}}   \newcommand{\sD}{{\mathscr{D}}}
\newcommand{\sE}{{\mathscr{E}}}   \newcommand{\sF}{{\mathscr{F}}}
\newcommand{\sG}{{\mathscr{G}}}   \newcommand{\sH}{{\mathscr{H}}}
\newcommand{\sI}{{\mathscr{I}}}   \newcommand{\sJ}{{\mathscr{J}}}
\newcommand{\sK}{{\mathscr{K}}}   \newcommand{\sL}{{\mathscr{L}}}
\newcommand{\sM}{{\mathscr{M}}}   \newcommand{\sN}{{\mathscr{N}}}
\newcommand{\sO}{{\mathscr{O}}}   \newcommand{\sP}{{\mathscr{P}}}
\newcommand{\sQ}{{\mathscr{Q}}}   \newcommand{\sR}{{\mathscr{R}}}
\newcommand{\sS}{{\mathscr{S}}}   \newcommand{\sT}{{\mathscr{T}}}
\newcommand{\sU}{{\mathscr{U}}}   \newcommand{\sV}{{\mathscr{V}}}
\newcommand{\sW}{{\mathscr{W}}}   \newcommand{\sX}{{\mathscr{X}}}
\newcommand{\sY}{{\mathscr{Y}}}   \newcommand{\sZ}{{\mathscr{Z}}}


%---------------------------------------
% Math Fraktur Font
%---------------------------------------

%Captital Letters
\newcommand{\mfA}{\mathfrak{A}}	\newcommand{\mfB}{\mathfrak{B}}
\newcommand{\mfC}{\mathfrak{C}}	\newcommand{\mfD}{\mathfrak{D}}
\newcommand{\mfE}{\mathfrak{E}}	\newcommand{\mfF}{\mathfrak{F}}
\newcommand{\mfG}{\mathfrak{G}}	\newcommand{\mfH}{\mathfrak{H}}
\newcommand{\mfI}{\mathfrak{I}}	\newcommand{\mfJ}{\mathfrak{J}}
\newcommand{\mfK}{\mathfrak{K}}	\newcommand{\mfL}{\mathfrak{L}}
\newcommand{\mfM}{\mathfrak{M}}	\newcommand{\mfN}{\mathfrak{N}}
\newcommand{\mfO}{\mathfrak{O}}	\newcommand{\mfP}{\mathfrak{P}}
\newcommand{\mfQ}{\mathfrak{Q}}	\newcommand{\mfR}{\mathfrak{R}}
\newcommand{\mfS}{\mathfrak{S}}	\newcommand{\mfT}{\mathfrak{T}}
\newcommand{\mfU}{\mathfrak{U}}	\newcommand{\mfV}{\mathfrak{V}}
\newcommand{\mfW}{\mathfrak{W}}	\newcommand{\mfX}{\mathfrak{X}}
\newcommand{\mfY}{\mathfrak{Y}}	\newcommand{\mfZ}{\mathfrak{Z}}
%Small Letters
\newcommand{\mfa}{\mathfrak{a}}	\newcommand{\mfb}{\mathfrak{b}}
\newcommand{\mfc}{\mathfrak{c}}	\newcommand{\mfd}{\mathfrak{d}}
\newcommand{\mfe}{\mathfrak{e}}	\newcommand{\mff}{\mathfrak{f}}
\newcommand{\mfg}{\mathfrak{g}}	\newcommand{\mfh}{\mathfrak{h}}
\newcommand{\mfi}{\mathfrak{i}}	\newcommand{\mfj}{\mathfrak{j}}
\newcommand{\mfk}{\mathfrak{k}}	\newcommand{\mfl}{\mathfrak{l}}
\newcommand{\mfm}{\mathfrak{m}}	\newcommand{\mfn}{\mathfrak{n}}
\newcommand{\mfo}{\mathfrak{o}}	\newcommand{\mfp}{\mathfrak{p}}
\newcommand{\mfq}{\mathfrak{q}}	\newcommand{\mfr}{\mathfrak{r}}
\newcommand{\mfs}{\mathfrak{s}}	\newcommand{\mft}{\mathfrak{t}}
\newcommand{\mfu}{\mathfrak{u}}	\newcommand{\mfv}{\mathfrak{v}}
\newcommand{\mfw}{\mathfrak{w}}	\newcommand{\mfx}{\mathfrak{x}}
\newcommand{\mfy}{\mathfrak{y}}	\newcommand{\mfz}{\mathfrak{z}}


\DeclarePairedDelimiter\pars{\lparen}{\rparen}

\begin{document}

\begin{titlepage}
	\centering
	\vspace*{2.5cm}

	{\sffamily \textbf{\Huge Definitionen und Sätze der HM 1 \& 2}\par}
	\vspace{2cm}
	{\huge \textsc{Julian Molt}\par}
	\vspace{1cm}
	{\Large \today\par}

	\vspace{\fill}

	{\scalebox{6}{\(\displaystyle \int\limits_{\text{\twemoji{snowflake}}}^{\text{\twemoji{sun}}}  \d s\)}}

	\vspace{\fill}

	{Aus den Vorlesungen von Prof. Dr. \textsc{Timo Weidl} in WS 2024 \& SS 2025 an der Universität Stuttgart.}
\end{titlepage}


% or \cleardoublepage
\newpage% \pdfbookmark[<level>]{<title>}{<dest>}
\pdfbookmark[section]{\contentsname}{toc}
\tableofcontents
\pagebreak

\chapter{Grundlagen der Mathematik}
\section{Elementare Logik}
\section{Naive Mengenlehre}
\section{Relationen und Funktionen}
\section{Die Zahlenbereiche}
\section{Die komplexen Zahlen}
\section{Zur Faktorisierung von Polynomen}

\thm{Hauptsatz der Algebra}{
Jedes Polynom über \(\CC\) vom Grad \(\deg{P} > 1\) besitzt mindestens eine Nullstelle \(z \in \CC\) (in der komplexen Ebene).
}
\section{Anwendungen}

\chapter{Grundlagen der Analysis}
\section{\texorpdfstring{Grenzwerte in \(\QQ\), Vollständigkeit}{Grenzwerte in Q, Vollständigkeit}}

\dfn{}{
Eine Folge \(a\) aus \(A\) ist eine Funktion \(a \colon \NN \to A\).\\
Man schreibt:
\begin{align*}
	a(1) &= a_1 \in A,\ldots,\\
	a(k) &= a_k \in A,\ldots,\\
	a &= \Le(a_k\Ri)_{k=1}^{\infty} = \Le(a_1, a_2, a_3, \ldots\Ri)
\end{align*}

\begin{itemize}
	\item Gleiche Werte können mehrfach angenommen werden.
	\item Die Anordnung ist wichtig.
\end{itemize}
}

\dfn{Grenzwert}{
Man nennt \(r \in \QQ\) Grenzwert einer Folge rationaler Zahlen \(\Le(a_n\Ri)_{n\in\NN}\) genau dann, wenn
\[\forall_{\varepsilon > 0} \enspace \exists_{N_{\varepsilon}\in\NN} \enspace \forall_{n\geqslant N_{\varepsilon}} \colon \enspace d\Le(a_n,r\Ri) < \varepsilon\]
Man schreibt dann
\[r = \lim_{n\to\infty} a_n \text{ oder kurz } a_n \overset{n\to\infty}{\longrightarrow} r\]
Eine Folge ist \underline{konvergent}, wenn sie einen Grenzwert besitzt.\\
Eine Folge ist \underline{divergent}, wenn sie keinen Grenzwert besitzt.
}

\dfn{}{
Eine Folge rationaler Zahlen \(a = \Le(a_n\Ri)_{n=1}^{\infty}\) ist beschränkt, genau dann, wenn
\[\exists_{C>0} \enspace \forall_{n\in\NN} \colon \enspace \abs{a_n} \leqslant C\]
}

\thm{}{Jede konvergente Folge ist beschränkt.}

\thm{}{
Wenn eine Folge \(\Le(a_n\Ri)_{n=1}^{\infty}\) gegen \(r\) konvergiert, dann konvergiert jede Teilfolge von \(a\) gegen denselben Grenzwert \(r\).
}

\thm{}{
Wenn eine Folge \(\Le(a_k\Ri)_{k=1}^{\infty}\) konvergiert, dann ist der Grenzwert eindeutig bestimmt.
}

\dfn{\textsc{Cauchy}-Folge, Fundamentalfolge}{
\(\Le(a_n\Ri)_{n=1}^{\infty}\) ist eine \textsc{Cauchy}-Folge, genau dann, wenn
\[\forall_{\varepsilon > 0} \enspace \exists_{N_{\varepsilon} \in \NN} \enspace \forall_{n,m \geqslant N_{\varepsilon}} \colon \enspace d(a_n,a_m) < \varepsilon\]
in \(\QQ \enspace d(a_n,a_m) = \abs{a_n-a_m}\)
}

\thm{}{Jede konvergente Folge ist eine \textsc{Cauchy}-Folge.}

\section{Die reellen Zahlen}

\dfn{Grundrechenarten auf \(\RR\)}{
\(r,s \in \RR\)\vspace{3pt}\\
\(r = \Le[\Le(r_k\Ri)_{k=1}^{\infty}\Ri] \quad \Le(r_k\Ri)_{k=1}^{\infty} \in \CauchySeq(\QQ)\)\vspace{3pt}\\
\(r = \Le[\Le(s_k\Ri)_{k=1}^{\infty}\Ri] \quad \Le(s_k\Ri)_{k=1}^{\infty} \in \CauchySeq(\QQ)\)\\
\begin{align*}
	r + s &\eqdef \Le[\Le(r_k + s_k\Ri)_{k=1}^{\infty}\Ri]_{\sim}\\
	r \cdot s &\eqdef \Le[\Le(r_k \cdot s_k\Ri)_{k=1}^{\infty}\Ri]_{\sim}
\end{align*}
}

\dfn{Ordnung auf \(\RR\)}{
\(r,s\) stehen für approximierte Folgen.\\
\((r_k) \in r\)\\
\((s_k) \in s\)
\[r < s \quad \overset{\text{\tiny{def}}}{\Longleftrightarrow} \quad \exists_{p,q \in \QQ \colon p < q} \enspace \exists_{N \in \NN} \enspace \forall_{n \geqslant N} \colon \enspace r_n \leqslant p \leqslant q \leqslant s_n\]
}

\thm{}{
Für beliebige \(r,s \in \RR\) gilt immer genau einer der folgenden Fälle:
\begin{align*}
	r &< s\\
	r &= s\\
	r &> s
\end{align*}
}

%Maybe Absolutbetrag und Abstsand auf R auch eine Definition - Nein

\section{\texorpdfstring{Grenzwerte in \(\RR\)}{Grenzwerte in R}}

\dfn{Monotonie}{
\((a_n), a_n \in \RR, n \in \NN\)\\
Monoton wachsend \((a_n)\!\uparrow\)
\[a_n \leqslant a_{n+1} \text{ für alle } n \in \NN\]
Streng monoton wachsend \((a_n)\!\uparrow\uparrow\)
\[a_n < a_{n+1} \text{ für alle } n \in \NN\]
Monoton fallend  \((a_n)\!\downarrow\)
\[a_n \geqslant a_{n+1} \text{ für alle } n \in \NN\]
Streng monoton fallend \((a_n)\!\downarrow\downarrow\)
\[a_n > a_{n+1} \text{ für alle } n \in \NN\]
}

\thm{}{Jede monotone beschränkte Folge reeller Zahlen besitzt einen Grenzwert in \(\RR\)}

\section{Maximum, Minimum, Infimum, Supremum}

\dfn{Maximum, Minimum}{
Sei \(M \sbst \RR\). Wir sagen \(a \in \RR\) ist das\\
Maximum von \(M\)
\[(a = \max{M}) \overset{\text{\tiny{def}}}{\Longleftrightarrow} (a \in M) \wedge (\forall_{x \in M} \colon \enspace x \leqslant a)\]
Minimum von \(M\)
\[(a = \min{M}) \overset{\text{\tiny{def}}}{\Longleftrightarrow} (a \in M) \wedge (\forall_{x \in M} \colon \enspace x \geqslant a)\]
}

\dfn{}{
Eine Menge \(M \in \RR\) ist beschränkt, wenn \(C > 0\) existiert, sodass
\[\abs{x} \leqslant C \text{ für alle } x \in M\]
}

\dfn{}{
Sei \(M \in \RR, M \neq \emptyset\).\\
Menge der oberen Schranken von \(M\)
\[M_+ = \Le\{y \in \RR \colon \forall_{x \in M} \colon x \leqslant y\Ri\}\]
Menge der unteren Schranken von \(M\)
\[M_- = \Le\{y \in \RR \colon \forall_{x \in M} \colon y \leqslant x\Ri\}\]
}

\dfn{}{
Für \(M \sbst \RR\) nennen wir \(a \in \RR\) das Supremum von \(M \Leftrightarrow a = \sup{M} = \min{M_+}\), bzw. das Infimum von \(M \Leftrightarrow a = \inf{M} = \max{M_-}\)\\
Falls \(\inf{M}\) bzw. \(\sup{M}\) existieren, dann sind diese eindeutig bestimmt.
}

\thm{Satz von Supremum und Infimum}{
Ist \(M \neq \emptyset\) nach oben beschränkt, dann existiert \(a = \sup{M}\).\\
Ist \(M \neq \emptyset\) nach unten beschränkt, dann existiert \(a = \inf{M}\).
}

\section{Die Zahl e}

\thm{}{Die Folge \(\Le(x_n\Ri)\) konvergiert in \(\RR\).}

\dfn{}{\[\e \coloneq \lim_{n \to \infty} x_n\]}

\thm{}{\[x_n < \e < x_n + \frac{1}{n!} \cdot \frac{1}{n} \qquad n \in \NN\]}

\thm{}{\(\e\) ist irrational.}

\thm{}{\[\e = \lim_{n\to\infty} \Le(1 + \frac{1}{n}\Ri)^n\]}

\section{Reihen}
\dfn{}{
Man sagt, dass die Reihe \(\sum_{k=1}^{\infty} a_k\) konvergiert, falls \(\Le(S_n\Ri)_{n=1}^{\infty}\) konvergiert.\\
Man setzt dann
\[\sum_{k=1}^{\infty} a_k = \lim_{n\to\infty} S_n\]
Sonst divergiert die Reihe \(\sum_{k=1}^{\infty} a_k\).
}

\thm{Doppelreihen}{
\(a_{m,n} > 0 \qquad m,n \in \NN\)\\
Folgende Reihen konvergieren gleichzeitig und sind gleich:
\[\sum_{(m,n)\in \NN \times \NN} = \sum_{m=1}^{\infty}\Le(\sum_{n=1}^{\infty} a_{m,n}\Ri) = \sum_{n=1}^{\infty}\Le(\sum_{m=1}^{\infty} a_{m,n}\Ri)\]
}

\dfn{}{Die Reihe \(\sum_{k=1}^{\infty} a_k\) konvergiert absolut, falls
\[\sum_{k=1}^{\infty} \abs{a_k}\]
konvergiert.
}

\thm{}{Jede absolut konvergente Reihe konvergiert.}

\thm{Umordungssatz für absolut konvergente Reihen}{
\(a_k \in \RR, \enspace k \in \NN\)\\
\(\sum_{k=1}^{\infty} a_k\) konvergiere absolut\\
\(\Phi \colon \NN \overset{\text{bij.}}{\longrightarrow} \NN \quad b_k = a_{\Phi(k)}\)
\[\implies \quad \sum_{k=1}^{\infty} b_k \text{ konvergiert absolut und } \sum_{k=1}^{\infty} a_k = \sum_{k=1}^{\infty} b_k.\]
}

\thm{Umordnungssatz von \textsc{Riemann}}{
\(a_k = a_k^+ - a_k^-\)\\
\(a_k \overset{k \to \infty}{\longrightarrow} 0\)\\
Angenommen beide Reihen \(\sum_{k=1}^{\infty} a_k^+\) und \(\sum_{k=1}^{\infty} a_k^-\) divergieren. Dann existiert für jedes \(r \in \RR\) eine Umordnung \(\Phi \colon \NN \overset{\text{bij.}}{\longrightarrow} \NN\), sodass
\[\sum_{k=1}^{\infty} a_{\Phi_r(k)} = r \,.\]
}

\thm{}{
\(\prod_{k=1}^{\infty} a_k\) konvergiert genau dann, wenn die Reihe \(\sum_{k=1}^{\infty} \ln{(a_k)}\) konvergiert, wobei
\[\ln{\Le(\prod_{k=1}^{\infty} a_k\Ri) = \sum_{k=1}^{\infty} \ln{(a_k)}}\]
}

\dfn{}{
Eine Reihe \(\sum_{k=1}^{\infty} a_k\) konvergiert nach \textsc{Cesaro} gegen \(S\) genau dann, wenn
\[S = \lim_{n\to\infty} \frac{1}{n}\Le(S_1 + \ldots + S_k\Ri)\]
}

\section{\texorpdfstring{Zur Struktur der Räume \(\mathbb{R}^n\) und \(\mathbb{C}^n\)}{Zur Struktur der Räume Rⁿ und Cⁿ}}

\dfn{}{
Eine Menge \(V\) nennt man Vektorraum über den Körper \(\KK\), falls die Operationen
\begin{align*}
	+ &\colon V \times V \to V\\
	\cdot \:&\colon \KK \times V \to V
\end{align*}
existieren mit folgenden Eigenschaften:
\begin{enumerate}
	\item \((V,+)\) ist eine \textsc{Abel}sche Gruppe:
	\begin{itemize}
		\item \(\Le(A_+\Ri): (x + y) + z \enspace=\enspace x + (y + z)\)
		\item \(\Le(N_+\Ri): \exists_{0_V \in V} \enspace \forall_{x \in V} \colon \enspace x + 0_V \enspace=\enspace 0_V + x \enspace=\enspace x\)
		\item \(\Le(I_+\Ri): \forall_{x \in V} \enspace \exists_{(-x) \in V} \colon \enspace x + (-x) \enspace=\enspace (-x) + x \enspace=\enspace 0_V\)
		\item \(\Le(K_+\Ri): x + y \enspace=\enspace y + x \text{ mit } x,y,z \in V\)
	\end{itemize}
	\item Eigenschaften der Multiplikation mit einem Skalar:
	\begin{enumerate}[label=(S\arabic*)]
		\item \(\alpha \cdot (x+y) = \alpha \cdot x + \alpha \cdot y\)
		\item \((\alpha + \beta) \cdot x = \alpha \cdot x + \beta \cdot x\)
		\item \(\alpha \cdot (\beta \cdot x) = (\alpha \cdot \beta) \cdot x\)
		\item \(1_{\KK} \cdot x = x\)
	\end{enumerate}
	für alle \(\alpha, \beta \in \KK, \enspace x, y \in V\)
\end{enumerate}

}

\dfn{Reelles Skalarprodukt}{
Sei \(V\) ein Vektorraum über \(\KK = \RR\). Dann nennt man \(\langle\cdot,\cdot\rangle\colon V \times V \to \RR\) mit den Eigenschaften \((1)_{S\RR}\text{--}(3)_{S\RR}\) ein (reelles) Skalarprodukt auf \(V\).
}

\dfn{Komplexes Skalarprodukt}{
	Sei \(V\) ein Vektorraum über \(\KK = \CC\). Dann nennt man \(\langle\cdot,\cdot\rangle\colon V \times V \to \CC\) mit den Eigenschaften \((1)_{S\CC}\text{--}(3)_{S\CC}\) ein (komplexes) Skalarprodukt auf \(V\).
}

\section{Metrische Räume}

\dfn{\(\varepsilon\)-Umgebung}{
\((M,d)\) metrischer Raum
\[U_{\varepsilon}(x) = \Le\{y \in M \colon d(x,y) < \varepsilon\Ri\} \qquad x \in M,\,\varepsilon > 0\]
}

\dfn{Grenzwert einer Folge}{
\(x_n \in M\)\\
\(y \in M\)\\
\((M,d)\) metrischer Raum
\[y = \lim_{n \to \infty} x_n \quad \overset{\text{\tiny{def}}}{\Longleftrightarrow} \quad \forall_{\varepsilon > 0} \enspace \exists_{N_{\varepsilon} \in \NN} \enspace \forall_{n \geqslant N_{\varepsilon}} \colon \enspace {\underbrace{d(x_n,y) < \varepsilon}_{x_n \in U_{\varepsilon}(y)}}\]
}

\thm{}{
Falls \((x_n)\) in \((M,d)\) konvergiert, so ist \(y = \lim_{n \to \infty} x_n\) eindeutig bestimmt.
}

\thm{}{
Jede konvergente Folge ist beschränkt, d.h. die Menge der Folgenglieder ist beschränkt.
}

\dfn{\textsc{Cauchy}-Folge, Fundamentalfolge}{
\((M,d)\) metrischer Raum\\
\(x_n \in M\)\\
\(n \in M\)
\[(x_n) \in \CauchySeq(M,d) \quad \overset{\text{\tiny{def}}}{\Longleftrightarrow} \quad \forall_{\varepsilon > 0} \enspace \exists_{N_{\varepsilon} \in \NN} \enspace \forall_{n,m \geqslant N_{\varepsilon}} \colon \enspace d(x_n,x_m) < \varepsilon \]
}

\thm{}{
Jede konvergente Folge ist eine Cauchy-Folge. Die Umkehrung gilt im Allgemeinen nicht.
}

\dfn{}{
Ein metrischer Raum \((M,d)\) ist vollständig, g.d.w. jede \textsc{Cauchy}-Folge einen Grenzwert in \(M\) besitzt.
}

\section{\texorpdfstring{Zur Topologie im \(\RR^n\) und \(\CC^n\)}{Zur Topologie im Rⁿ und Cⁿ}}

\dfn{Häufungspunkt einer Menge}{
\((M,d)\) metrischer Raum\\
Sei \(X \sbst M\). Wir nennen \(x_0 \in M\) Häufungspunkt von \(X\) g.d.w.\\
\[\forall_{\varepsilon > 0} \colon \enspace U_{\varepsilon}(x_0) \cap \Le(X \backslash \{x_0\}\Ri) \neq \emptyset\]
}

\dfn{Isolierter Punkt}{
Sei \(X \sbst M\). Wir nennen \(x_0 \in X\) einen isolierten Punkt von \(X\) g.d.w.
\[\exists_{\varepsilon > 0} \colon \enspace U_{\varepsilon}(x_0) \cap \Le(X \backslash \{x_0\}\Ri) = \emptyset\]
}

\dfn{}{
Wir nennen \(x_0 \in M\)\\
\begin{itemize}
    \item inneren Punkt von \(X\)\\[1ex]
    \(\exists_{\varepsilon > 0} \colon \enspace U_{\varepsilon}(x_0) \sbst X\)\\
    \item äußeren Punkt zu \(X\)\\[1ex]
    \(\exists_{\varepsilon > 0} \colon \enspace U_{\varepsilon}(x_0) \sbst (M \backslash X)\)\\
    \item Randpunkt von \(X\)\\[1ex]
    \(\forall_{\varepsilon > 0} \colon \enspace \Le(U_{\varepsilon}(x_0) \cap X \neq \emptyset\Ri) \wedge \Le(U_{\varepsilon}(x_0) \cap (M \backslash X) \neq \emptyset\Ri)\)
\end{itemize}

\(\interior{X}\) Menge der inneren Punkte\\
\(\exterior{X}\) Menge der äußeren Punkte\\
\(\edge{X}\) Menge der Randpunkte
}

\dfn{}{
Eine Menge \(X \sbst M\) heißt \underline{offen}, g.d.w.
\[X = \interior{X}\]
}

\dfn{}{
Eine Menge \(X \sbst M\) heißt \underline{abgeschlossen} g.d.w.
\[X = \interior{X} \cup \edge X\]
}

\thm{}{
\(X \sbst M\) ist offen in \((M,d)\), g.d.w. \(M \backslash X\) abgeschlossen in \((M,d)\) ist.
}

\section{\texorpdfstring{Die Exponentialfunktion -- Die Formel von \textsc{Euler}}{Die Exponentialfunktion - Die Formel von Euler}}
Sei im Folgenden
\[t_n(z) = 1 + \sum_{k=1}^{n} \frac{z^k}{k!} \quad, z \in \CC,\, n \in \NN\]

\thm{}{
Die Folge \((t_n(z))_{n\in\NN}\) ist für jedes \(z\in\CC\, n \to \infty\) konvergent.
}

\dfn{}{
\(\exp\colon \CC \to \CC\)
\[\exp(z) = \lim_{n \to \infty} t_n(z) = 1 + \sum_{k=1}^{\infty} \frac{z^k}{k!}\]
Bzw. mit den Vereinbarungen\\
\(0! = 1\), \(z^0 = 1\) (auch für \(z = 0\))
\[\exp(z) = \sum_{k=0}^{\infty} \frac{z^k}{k!}\]
}

\thm{}{
Für alle \(z, w \in \CC\) gilt:
\[\exp(z+w) = \exp(z) \cdot \exp(w)\]
}

\thm{}{
Für \(z \in \CC\) mit  \(z < 1\) gilt:
\[\abs{\exp(z) - 1 - z} \leqslant \abs{z}^2\]
bzw:
\[\exp(z) = 1 + z + R(z)\]
mit \(\abs{R(z)} \leqslant \abs{z}^2\) für \(\abs{z} < 1\)\\
(\(R(z)\) ist der Rest von \(z\))
}

\dfn{}{
\[\mathrm{e}^x = \exp(x) \qquad x \in \RR\]
}

\section{Grenzwerte von Funktionen}

\dfn{\(\varepsilon\)-\(\delta\)-Definition}{
\(\displaystyle y_0 = \lim_{x \to x_0} f(x)\) g.d.w.
\[\forall_{\varepsilon > 0} \enspace \exists_{\delta > 0} \colon \enspace f\Big(X \cap \overset{\circ}{U_{\delta}}(x_0)\Big) \sbst U_{\varepsilon}(y_0)\]
}

\dfn{Folgendefinition}{
\(\displaystyle y_0 = \lim_{x \to x_0} f(x)\) g.d.w. für \underline{jede} Folge \((x_n)_{n = 1}^{\infty}\) mit \(x_n \in X\backslash\{x_0\}\),
\(x_n \overset{n \to \infty}{\longrightarrow} x_0\) gilt:
\[y_0 = \lim_{n \to \infty} f(x_n)\]
}

\thm{}{Die letzten beiden Definitionen sind äquivalent zueinander.}

\section{Stetigkeit}

\dfn{}{
\(f\) ist im Punkt \(x_0\) stetig, g.d.w.
\begin{enumerate}
    \item \(x_0 \in \isolated(X)\) oder
    \item \(\lim_{x \to x_0} f(x) = f(x_0)\) für \(x_0 \in \accumulated(X)\)
\end{enumerate}
}

\dfn{\(\varepsilon\)-\(\delta\)-Definition}{
\(f \colon X \sbst M_1 \to M_2\) ist stetig in \(x_0 \in X \Leftrightarrow\)
\[\forall_{\varepsilon > 0} \enspace \exists_{\delta > 0} \colon \enspace f(X \cap U_{\delta}(x_0)) \sbst U_{\varepsilon}(f(x_0))\]
}

\dfn{Folgendefinition}{
\(f \colon X \sbst M_1 \to M_2\) stetig in \(x_0 \in X\)\\
\(\Leftrightarrow\) für jede Folge \((x_n)_{n \in \NN}\) mit \(x_n \in X\), \(x_n \overset{n \to \infty}{\longrightarrow} x_0\) gilt:
\[\lim_{n \to \infty} f(x_n) = f(x_0)\]
}

\dfn{}{
\(f \colon X \sbst M_1 \to M_2\) ist stetig auf \(X\), wenn \(f\) in jedem Punkt \(x_0 \in X\) stetig ist.
}

\thm{}{
Sei \(X = M_1\), dann ist \(f(X = M_1 \to M_2)\) stetig auf \(X = M_1\) g.d.w. das Urbild \(f^{-1}(U)\) von jeder in \(M_2\)
offenen Menge \(U \sbst U_2\) in \(M_2\) ist.
}

\section{Stetige reelle Funktionen einer reellen Variablen}

\thm{Satz von \textsc{Bolzano} und \textsc{Cauchy}}{
\(f \colon [a, b] \overset{\text{stetig}}{\longrightarrow} \RR\)\\
\(a < b,\, f(a) \cdot f(b) < 0\)
\[\Rightarrow \quad \exists_{C \in {]{a,b}[}} \colon \enspace f(C) = 0\]
}

\dfn{Monotonie}{
\begin{align*}
    &f \!\uparrow \text{(monoton wachsend)} &x_1 < x_2 \Rightarrow f(x_1) \leqslant f(x_2)\\
    &f \!\uparrow\uparrow \text{(streng monoton wachsend)} &x_1 < x_2 \Rightarrow f(x_1) < f(x_2)\\
    &f \!\downarrow \text{(monoton fallend)} &x_1 < x_2 \Rightarrow f(x_1) \geqslant f(x_2)\\
    &f \!\downarrow\downarrow \text{(streng monoton fallend)} &x_1 < x_2 \Rightarrow f(x_1) > f(x_2)\\
\end{align*}
}

\section{Kompaktheit}

\dfn{Häufungspunkt einer Folge}{
Wir nennen \(y \in M\) Häufungspunkt einer Folge \((a_n)_{n\in\NN}\), falls eine Teilfolge \((a_j)_{j\in\NN}\) existiert,
welche gegen \(y\) konvergiert.
}

\dfn{Kompaktheit}{
\((M,d)\) metrischer Raum, \(K \sbst M\)\\
\(K\) ist (folgen)kompakt, g.d.w. jede Folge aus \(K\) mindestens einen Häufungspunkt aus \(K\) enthält.
}

\thm{\textsc{Bolzano-Weierstraß}}{
\((M,d) = \Le(\bbK^m,\,d_{|\cdot|}\Ri)\)\vspace{3pt}\\
Für eine Menge \(K \sbst \bbK^m\) gilt Kompaktheit, g.d.w. Beschränktheit und Abgeschlossenheit gelten.
}

\thm{}{
Wenn \(K \sbst M_1\) kompakt \(f \colon K \sbst M_1 \to M_2\) auf \(K\) stetig ist, dann ist \(f(K)\) kompakt in \(M_2\)
}

\thm{\textsc{Weierstraß}}{
\(f \colon K \sbst M_1 \to \RR\) stetig und \(K \sbst M_1\) kompakt\\
\(\Rightarrow f\) ist beschränkt und nimmt einen Minimalwert und einen Maximalwert an.
}

\thm{}{
\(f \colon K \sbst M_1 \overset{\text{stetig}}{\longrightarrow} M_2\)\\
\(K \sbst M_1\) kompakt
\[\implies \quad \forall_{\varepsilon > 0} \enspace \exists_{\delta = \delta_{\varepsilon} > 0} \enspace \forall_{x_0 \in K}\]
}

\chapter{Zur Differenzialrechnung für Funktionen einer Var.}

\section{Differenzialquotient und Ableitung}

\dfn{Differenzialquotient}{\[
\varphi(f,x_0,h) = \frac{1}{h}\Le(f(x_0+h) - f(x_0)\Ri) \qquad h \neq 0
\]}

\dfn{}{
\(f\) ist im Punkt \(x_0\) differenzierbar g.d.w.
\[\lim_{h \to 0} \varphi(f,x_0,h) = F \in \mathbb{R}\] existiert.
}

\dfn{}{
Wir nennen \(f\) in \(z_0\) differenzierbar g.d.w.
\[\lim_{h \to 0} \varphi(f,z_0,h) = f\prime(z_0) \in \mathbb{C}\] existiert.
}

\section{\texorpdfstring{Die Landau-Symbole \(\mathcal{o}\) und \(\mathcal{O}\)}{Die Landau-Symbole o und O}}

\dfn{\textsc{Landau}-Symbole}{
\(f \overset{x \to x_0}{=} \mathcal{O}(g)\) g.d.w.
\[\exists_{\delta>0} \enspace \exists_{C\in \mathbb{R}} \enspace \forall_{x \in U_{\delta}(x_0) \cap X} \colon \enspace \norm{f(x)} \leqslant C \cdot \abs{g(x)}\]
\(f \overset{x \to x_0}{=} \mathcal{o}(g)\) g.d.w
\[\forall_{\varepsilon>0} \enspace \exists_{\delta_{\varepsilon}>0} \enspace \forall_{x\in U_{\delta_{\varepsilon}}(x_0)\cap X} \colon \enspace \norm{f(x)} \leqslant \varepsilon \cdot \abs{g(x)}\]
}

\section{Regeln für das Rechnen mit Ableitungen}

\(f,f_1,f_2\) wie eben\\
\(g \colon X \to \KK\)

\thm{}{
\(f, f_1, f_2, g\) differenzierbar im Punkt \(x_0 \in \interior(X)\)\\
\(\implies\) Dann existieren folgende Ableitungen in \(x_0\):
\begin{enumerate}
    \item \((f_1 \pm f_2)\prime(x_0) = f_1\prime(x_0) \pm f_2\prime(x_0)\)
    \item \((\alpha \cdot f)\prime(x_0) = \alpha \cdot f\prime(x_0) \qquad \alpha \in \bbK\)
    \item \((g \cdot f)\prime(x_0) = g\prime(x_0) \cdot f(x_0) + g(x_0) \cdot f\prime(x_0)\)
\end{enumerate}
}

\thm{Kettenregel}{
\((f \circ \psi)(y) = f\Le(\psi(y)\Ri)\)
\[\eval{\odv[fun=false]{(f \circ \psi)}{y}}_{y=y_0}
= \eval{\odv{\psi}{y}}_{y=y_0} \cdot \eval{\odv{f}{x}}_{x=x_0=\psi(y_0)}\]
}

\thm{Quotientenregel}{
\(f,\, g \colon X \to \bbK;\, x_0 \in \interior(X)\)\\
\(g(x) \neq 0\) für \(x \in X\)\\
\(f\) und \(g\) sind in \(x_0\) differenzierbar
\[\implies \quad \eval{\odv{}{x}\Le(\frac{f}{g}\Ri)}_{x=x_0} = \frac{f\prime(x_0) \cdot g(x_0) - f(x_0) \cdot g\prime(x_0)}{\Le(g(x_0)\Ri)^2}\]
}

\thm{Ableitung der Umkehrfunktion}{
\(f \colon X \to Y\) bijektiv\\
\(x_0 \in \interior(X)\)\\
\(y_0 \in \interior(Y)\)\\
\(f\) in \(x_0\) differenzierbar; \(f\prime(x_0) \neq 0\)\\
\(f^{-1}\) ist in \(y_0 = f(x_0)\) stetig
\[\implies \quad \eval{\odv{f^{-1}}{y}}_{y=y_0} = \frac{1}{\eval{\odv{f}{x}}_{x=x_0}}\]
}

\section{\texorpdfstring{Die Sätze von \textsc{Fermat, Rolle}: Die Formel von \textsc{Cauchy} und \textsc{Lagrange}}{Die Sätze von Fermat, Rolle: Die Formel von Cauchy und Lagrange}}

\thm{\textsc{Fermat}}{
\(f\colon[a,b] \to \RR\)\\
\(a < c < b\), \(f\) ist in \(c\) differenzierbar\\
\(\displaystyle f(c) = \max_{x \in [a,b]} f(x)\) oder\\
\(\displaystyle f(c) = \min_{x \in [a,b]} f(x)\)
\[\Rightarrow \quad f\prime(c) = 0\]
}

\thm{\textsc{Rolle}}{
\(f\colon[a,b] \overset{\text{stetig}}{\longrightarrow} \RR\)\\
\(a < b\)\\
\(f\) differenzierbar in \(]a,b[\)\\
\(f(a) = f(b)\)
\[\Rightarrow \quad \exists_{c \in {]{a,b}[}} \colon \enspace f\prime(c) = 0\]
}

\thm{\textsc{Cauchy}}{ %Cauchy's Mean Value Theorem
\(a < b\)\\
\(f,g \colon [a,b] \to \RR\) stetig\\
\(f,g\) auf \(]a,b[\) differenzierbar\\
\(g\prime(x) \neq 0\) für \(]a,b[\)\\
\[\implies \quad \exists_{c \in {]{a,b}[}} \colon \enspace \frac{f\prime(c)}{g\prime(c)} = \frac{f(b) - f(a)}{g(b) - g(a)}\]
}

\thm{Mittelwertsatz der Differenzialrechnung (Formel von \textsc{Lagrange})}{
\(f \colon {[{a,b}]} \overset{\text{stetig}}{\longrightarrow} \RR\), in \({]{a,b}[}\) differenzierbar
\[\implies \quad \exists_{c \in {]{a,b}[}} \colon \enspace f\prime(c) = \frac{f(b) - f(a)}{b - a}\]
}

\section{Der Hauptsatz der Differenzialrechnung}

\thm{Hauptsatz der Differenzialrechnung}{
\(\bbK = \RR\) oder \(\bbK = \CC\), \(n \in \NN\)\\
\(f \colon [a,b] \overset{\mathrm{stetig}}{\longrightarrow} \KK^n\) in \(\openint{a}{b}\) differenzierbar.
\[\implies \quad \norm{f(b) - f(a)} \leqslant \sup_{x\in\openint{a}{b}} \norm{f\prime(x)} \cdot \abs{b-a}\]
}

\section{Höhere Ableitungen}

\thm{Satz von \textsc{Leibniz}}{
	\(f \colon X \to \KK^m\)\\
	\(g \colon X \to \KK\)\\
	\(X\) offen, \(x_0 \in X\), \(f\) und \(g\) sind in \(x_0\) \(m\)-fach differenzierbar.\\
	Dann existiert
	\[(g \cdot f)^{(m)}(x_0) = \sum_{k=0}^{m} \binom{m}{k} \cdot g^{(m-k)}(x_0) \cdot f^{(k)}(x_0) \,.\]
}

\section{\texorpdfstring{Der Satz von \textsc{Taylor}}{Der Satz von Taylor}}

\thm{\textsc{Taylor}}{
\(f \colon {]{a,b}[} = X \to \bbK^n\) bzw.\\
\(f \colon X \sbst \CC \to \CC^n\)\\
\(X\) offen, \(x_0 \in X\)\\
Sei \(f\) im Punkt \(x_0\) \(m\)-fach differenzierbar, dann gilt
\[f(x_0 + h) \overset{h \to 0}{=} {\underbrace{f(x_0) + \sum_{k=1}^{m} \frac{f^{(k)}(x_0)}{k!} h^k}_{=T_m(x_0,h)}} + o(h^m)\]
}

\section{Anwendungen: Monotonie und Extremwerte}

\thm{}{
\begin{enumerate}
    \item \(f \!\uparrow \quad \Leftrightarrow f\prime(x) \geqslant 0\) für alle \(x \in {]{a,b}[}\)
    \item \(f \!\uparrow\uparrow \quad \Leftrightarrow f\prime(x) > 0\) für alle \(x \in {]{a,b}[}\) und es gibt keine \(\alpha, \beta \in {]{a,b}[}\) mit \(\alpha < \beta\) und \(f\prime(x) = 0\) für alle \(x \in {]{\alpha,\beta}[}\)
\end{enumerate}
}

\section{Konvexität und Konkavität}

\dfn{}{
\(f \colon \openint{a}{b} \to \RR\) ist \underline{konvex} g.d.w. für alle \(a < x_1 < x_2 < b\) und alle \(t \in [0,1]\) gilt mit \(x(t) = t \cdot x_1 + (1-t) \cdot x_2\)
\[f(x(t)) \leqslant t \cdot f(x_1) + (1-t) \cdot f(x_2)\]
\(f\) ist \underline{konkav} g.d.w. \(-f\) konvex ist.
}

\thm{}{
Ist \(f \colon \openint{a}{b}\) konvex (bzw. konkav), dann ist \(f\) stetig.
}

\thm{}{
Sei \(f \colon \openint{a}{b} \to \RR\) differenzierbar
\begin{enumerate}
    \item \(f\) ist konvex \quad \(\Leftrightarrow \quad f\prime \uparrow\) auf \(\openint{a}{b}\)
    \item \(f\) ist konkav \quad \(\Leftrightarrow \quad f\prime \downarrow\) auf \(\openint{a}{b}\)
\end{enumerate}
}

\thm{}{
Sei \(f \colon \openint{a}{b} \to \RR\) 2-fach differenzierbar in \(\openint{a}{b}\)
\begin{enumerate}
    \item \(f\) ist konvex \quad \(\Leftrightarrow \quad f\dprime(x) \geqslant 0\) für alle \(x \in \openint{a}{b}\)
    \item \(f\) ist konkav \quad \(\Leftrightarrow \quad f\dprime(x) \leqslant 0\) für alle \(x \in \openint{a}{b}\)
\end{enumerate}
}

\thm{}{
\(f \colon \openint{a}{b}\) 2-fach differenzierbar in \(c \in \openint{a}{b}\) in \(c\) liegt Wendepunkt vor
\[\implies \quad f\dprime(c) = 0\]
}

\section{\texorpdfstring{Unbestimmtheiten vom Typ \(0/0\) bzw. \(\infty / \infty\)}{Unbestimmtheiten vom Typ 0/0 bzw. ∞/∞}}


\thm{\textsc{Bernoulli, l'Hospital}}{
\(f,g \colon {]{a,b}[} \to \RR\) in \(\RR\) differenzierbar\\
\(g(x) \neq 0\), \(g\prime(x) \neq 0\) für \(x \in {]{a,b}[}\)\\
\(\displaystyle \lim_{x \to a} f(x) = \lim_{x \to a} g(x) = 0\)\\
Es existiere \(\displaystyle \lim_{x \to a} \frac{f\prime(x)}{g\prime(x)} = A \in \RR\)
\[\implies \quad \lim_{x \to a} \frac{f(x)}{g(x)} = A\]
}

\thm{Unbestimmtheiten vom Typ \(\frac{\infty}{\infty}\)}{
\(f,g \colon {]{a,b}[} \overset{\text{db.}}{\longrightarrow} \RR\)\\
\(g\prime(x) \neq 0\)\\
\(\displaystyle \lim_{x \to a} f(x) = \lim_{x \to a} g(x) = \infty\)\\
Es existiere \(\displaystyle \lim_{x \to a} \frac{f\prime(x)}{g\prime(x)} = A \in \RR\)
\[\implies \quad \lim_{x \to a} \frac{f(x)}{g(x)} = A\]
}

\chapter{Integralrechnung}

\section{\texorpdfstring{Das \textsc{Riemann}-Integral}{Das Riemann-Integral}}

\dfn{\textsc{Riemann}-Integral}{
Wir nennen \(f\colon [a,b] \to \RR\) Riemann-integrierbar, falls ein \(I \in \RR\) existiert, sodass
\[\lim_{n \to \infty} \sum \Le(f; \delta^{(n)}; \Xi^{(n)}\Ri) = I\]
Man schreibt
\[I = \int_{a}^{b} f(x) \d x\]

}

\thm{Struktur des Raumes R[a,b]}{
\(f,g \in R[a,b]\)\\
\([c,d] \sbst [a,b]\)\\
\(\alpha \in \RR\)
\begin{enumerate}[label=(\arabic*)]
	\item \(f + g \in R[a,b]\)
	\item \(\alpha \cdot f \in R[a,b]\)
	\item \(\abs{f}_{[c,d]} \in R[a,b]\)
	\item \(\eval{f}_{[c,d]} \in R[c,d]\)
	\item \(f \cdot g \in R[a,b]\)
\end{enumerate}
}

\thm{}{
Ändert man \(f \in R[a,b]\) in endlich vielen Punkten ab, dann ist die neue Funktion ebenfalls Riemann-integrierbar.
}

\dfn{Erweiterung}{
\(f \colon \{a\} \to \RR\)
\[\int_{a}^{a} f(x) \d x \eqdef 0\]
}

\dfn{Erweiterung (gerichtetes Integral)}{
Sei \(a \leqslant b\)
\[\int_{a}^{b} f(x) \d x \eqdef -\int_{b}^{a} f(x) \d x\]
}

\section{\texorpdfstring{Wichtige Eigenschaften des \textsc{Riemann}-Integrals}{Wichtige Eigenschaften des Riemann-Integrals}}

\dfn{\textsc{Riemann}-Integral für komplexwertige Funktionen}{
\(f \colon [a,b] \to \CC\)\\
\(f(x) = f_R(x) + \mathrm{i} f_I(x)\)\\
mit\\
\(f_R(x) = \Re{f(x)}\)\\
\(f_I(x) = \Im{f(x)}\)\\% hier noch Leerzeile einfügen
\(f \in R[a,b]\)
\[\Leftrightarrow \quad f_R \in R[a,b] \wedge f_I \in R[a,b]\]
und es gilt:
\[\int_{a}^{b} f(x) \d x = \int_{a}^{b} f_R(x) \d x + \mathrm{i} \int_{a}^{b} f_I(x) \d x\]
}

\section{\texorpdfstring{Die Formel von \textsc{Newton} und \textsc{Leibniz} -- Die Stammfunktion}{Die Formel von Newton und Leibniz - Die Stammfunktion}}

\thm{\textsc{Newton, Leibniz}}{
\(F \colon [a,b] \to \RR, \enspace a < b\)
\begin{enumerate}[label=(\arabic*)]
	\item \(F\) stetig auf \([a,b]\)
	\item \(F\) differenzierbar auf \(\openint{a}{b}\)\\
		  \(f \colon [a,b] \to \RR\)
	\item \(f(x) = \begin{cases} 0 & x = a \, \vee \, x = b \\ F\prime(x) & x \in \openint{a}{b}\end{cases}\)
\end{enumerate}
Es sei \(f \in R[a,b]\)
Dann gilt
\[
	\int_{a}^{b} f(x) \d x = F(b) - F(a) = \eval{F(x)}_a^b
\]
}

\dfn{}{
\(F \colon [a,b] \to \RR\)\\
\(f \colon [a,b] \to \RR\)\\
Wir nennen \(F\) Stammfunktion von \(f\), falls (1), (2) erfüllt sind und \((3)'\) \(f(x) = F'(x), \enspace x \in [a,b]\)
}

\thm{Hauptsatz der Differenzial- und Integralrechnung}{
Wenn \(f \in R[a,b]\) eine Stammfunktion \(F\) besitzt, dann gilt
\[
	\int_{a}^{b} f(x) \d x = F(b) - F(a)
\]
}

\thm{\textsc{Darboux}}{
Sei \(F \colon [a,b] \to \RR\) differenzierbar in \(\openint{a}{b}\) und \(f(x) = F\prime(x)\) für \(x \in \openint{a}{b}\). Dann besitzt \(f\) keine Sprungstelle in \(\openint{a}{b}\)
}

\thm{Existenz einer Stammfunktion}{
\(f \colon [a,b] \to \RR\)\\
Sei \(f\) in \(\openint{a}{b}\) stetig und auf \([a,b]\) beschränkt.\\
Dann ist \((y \in [a,b])\)
\[
	F(y) = F(a) + \int_{a}^{y} f(x) \d x
\]
eine Stammfunktion von \(f\) auf \([a,b]\)
}

\section{Partielle Integration, Substitution der Integrationsvariablen}

\section{Zur Integration rationaler Funktionen}

\section{Die Mittelwertsätze der Integralrechnung}

\thm{Erster Mittelwertsatz}{
\(f,g \colon [a,b] \to \RR\)\\
\(f,g\) stetig auf \([a,b], \enspace g(x) \geqslant 0\) für \(x \in [a,b]\)
\[
	\implies \quad \exists_{\xi \in [a,b]} \colon \enspace \int_{a}^{b} f(x) \cdot g(x) \d x \enspace = \enspace f(\xi) \cdot \int_{a}^{b} g(x) \d x
\]
}

%\thm{Zweiter Mittelwertsatz}{
%\(f,g \colon [a,b] \to \RR\)\\
%\(g \in R[a,b]\)
%\begin{enumerate}[label=(\arabic*)]
%	\item \(f \!\downarrow\) und \(f(x) \geqslant 0, \enspace x \in [a,b]\)\\
%		 \[ \begin{aligned}&\implies \quad &&\exists_{\xi \in [a,b]} \colon \enspace \int_{a}^{b} f(x) \cdot g(x) \d x \enspace = \enspace f(a) \cdot \int_{a}^{\xi} g(x) \d x\end{aligned}\]
%	\item \(f \!\uparrow\) und \(f(x) \leqslant 0, \enspace x \in [a,b]\)\\
%		  \[\begin{aligned}&\implies \quad &&\exists_{\xi \in [a,b]} \colon \int_{a}^{b} f(x) \cdot g(x) \d x \enspace = \enspace f(b) \cdot \int_{\xi}^{b} g(x) \d x\end{aligned}\]
%	\item \(f\) monoton
%		  \[\begin{aligned}&\implies \quad &&\exists_{\xi \in [a,b]} \colon \enspace \int_{a}^{b} f(x) \cdot g(x) \d x \enspace = \enspace f(a) \cdot \int_{a}^{\xi} g(x) \d x + f(b) \cdot \int_{\xi}^{b} g(x) \d x\end{aligned}\]
%\end{enumerate} %to do: Implikationspfeile alignen
%}

\thm{Zweiter Mittelwertsatz}{
\(f,g \colon [a,b] \to \mathbb{R}\)\\
\(g \in R[a,b]\)

\begin{enumerate}[label=(\arabic*)]
	\item \(f \!\downarrow\) und \(f(x) \geqslant 0, \enspace x \in [a,b]\)\\[1ex]
	\makebox[0pt][l]{\phantom{XXX}}% zum horizontalen Ausrichten
	\hspace{3.7em}%
	\(\displaystyle\implies \quad \exists_{\xi \in [a,b]} \colon \int_{a}^{b} f(x) \cdot g(x) \,\mathrm{d}x = f(a) \cdot \int_{a}^{\xi} g(x) \,\mathrm{d}x\)

	\item \(f \!\uparrow\) und \(f(x) \leqslant 0, \enspace x \in [a,b]\)\\[1ex]
	\makebox[0pt][l]{\phantom{XXX}}%
	\hspace{3.7em}%
	\(\displaystyle\implies \quad \exists_{\xi \in [a,b]} \colon \int_{a}^{b} f(x) \cdot g(x) \,\mathrm{d}x = f(b) \cdot \int_{\xi}^{b} g(x) \,\mathrm{d}x\)

	\item \(f\) monoton\\[1ex]
	\makebox[0pt][l]{\phantom{XXX}}%
	\hspace{3.7em}%
	\(\displaystyle\implies \quad \exists_{\xi \in [a,b]} \colon \int_{a}^{b} f(x) \cdot g(x) \,\mathrm{d}x = f(a) \cdot \int_{a}^{\xi} g(x) \,\mathrm{d}x + f(b) \cdot \int_{\xi}^{b} g(x) \,\mathrm{d}x\)
\end{enumerate}
}

\section{\texorpdfstring{Das Restglied in der Formel von \textsc{Taylor}}{Das Restglied in der Formel von Taylor}}

\thm{}{
Sei \(f\) in \(I_h(x_0) \, (m+1)\)-fach differenzierbar und \(f^{(m+1)}\) sei stetig auf \(I_h(x_0)\)
\[
	\implies \quad r_m(x_0,h) = \frac{h^{m+1}}{m!} \cdot \int_{0}^{1} f^{(m+1)}(x_0 + th)(1-t)^m \d t
\]
}

\section{Numerische Verfahren der Integration}

\thm{}{
Sei \(f \colon [a,b] \to \RR\)
\begin{itemize}
	\item in \(\openint{a}{b} \enspace (n+1)\)-fach differenzierbar
	\item \(f,f\prime,\hdots,f^{(n)},f^{(n+1)}\) stetig und stetig auf \([a,b]\) fortsetzbar.
\end{itemize}
Dann existiert zu jedem \(x \in [a,b]\) (mindestens) einen Punkt \(\xi \in [a,b]\) mit
\[f(x)-P_n(x) = \frac{f^{(n+1)}(\xi_x)}{(n+1)!}\cdot (x-x_0) \cdot \hdots \cdot (x-x_n)\]
}

\section{Einige Anwendungen der Differenzial- und Integralrechnung}

\dfnc{}{
\(\varphi\) erzeugt eine Kurve der Klasse \(C^p, p \in \NN\), falls zudem
\begin{enumerate}[label=(\arabic*)]
	\item \(\varphi\) \(p\)-fach stetig differenzierbar, Ableitungen stetig in Randpunkt fortsetzbar.
	\item \(\dot{\varphi}(t) \neq 0\) für \(t \in \openint{a}{b}\)\\
	      \[\exists \lim_{t \to a,b} \dot{\varphi} \neq 0\]
\end{enumerate}
Wobei \(\displaystyle \dot{\varphi}(t) = \odv{\varphi(t)}{t}\)
}

\dfn{}{
\(\varphi\) erzeugt eine rektifizierbare Kurve der Länge \(L\) g.d.w.
\[\sup_{\delta} l(\delta) = L < \infty.\]
}

\thm{}{
Die Abbildung \(\varphi \colon [a,b] \to \RR^n\) erzeugt eine Kurve der Klasse \(C^1\). Dann
\begin{enumerate}[label=(\arabic*)]
	\item erzeugt \(\varphi\) eine rektifizierbare Kurve.
	\item \(\displaystyle L = \int_{a}^{b} \norm{\dot{\varphi}(t)} \d t\)
\end{enumerate}
}

\dfn{}{
\(K(s) = \norm{\kappa(s)}\) Krümmung\\
\(R(s) = \frac{1}{K(s)}\) Krümmungsradius
}

\section{Flächen, Volumina}

\dfn{}{
Wir nennen \(\Omega\) quadrierbar, falls \(S_*(\Omega) = S^*(\Omega)\) und setzen \(A(\Omega) = S_*(\Omega) = S^*(\Omega)\)
}

\thm{}{
\(f \colon [a,b] \to \RR\) stetig\\
\(a \leqslant b, f(x) \geqslant 0\) für \(x \in [a,b]\)\\
\(\Omega = \Le\{(x,y) \in \RR^2 \vert (a \leqslant x \leqslant b) \wedge (0 \leqslant y \leqslant f(x))\Ri\}\)\\
\(\implies \Omega\) quadrierbar
\[A(\Omega) = \int_{a}^{b} f(x) \d x\]
}

\thm{}{
\(0 \leqslant \varphi \leqslant \beta \leqslant 2\pi\)\\
\((r,\varphi)\) Polarkoordinaten in \(\RR^2\)\\
\(f(\varphi) \geqslant 0\) für \(\varphi \in [\alpha,\beta]\)\\
\(\Omega = \Le\{(r,\varphi) \vert 0 \leqslant r \leqslant f(\varphi) \wedge \varphi \in [\alpha,\beta]\Ri\}\)\\
\(f\colon [\alpha,\beta] \to \RR\) stetig
\[\implies \quad A(\Omega) = \frac{1}{2} \int_{\alpha}^{\beta} \Le(f(\varphi)\Ri)^2 \d \varphi\]
}

\chapter{Lineare Algebra}

\section{\texorpdfstring{Matrizen -- Grundlagen}{Matrizen - Grundlagen}}

\dfn{}{
Eine Matrix vom Typ \((m,n)\) ist ein rechteckiges Schema von Zahlen aus \(\KK\), wobei \(\KK = \RR\) oder \(\KK = \CC\).
\begin{align*}
	A &=
	\begin{pmatrix}
	a_{11} & \cdots & a_{1n}\\
	\vdots & \ddots & \vdots\\
	a_{m1} & \cdots & a_{mn}
	\end{pmatrix} \quad m,n \in \NN\\[2ex]
	&= 	\Le(a_{ij}\Ri)_{j = 1,\ldots,m}^{i = 1,\ldots,n}
\end{align*}
}

\section{Quadratische Matrizen}

\dfn{}{
\begin{align*}
	[A,B] &= AB - BA \quad\text{ (Kommutator)}\\
	\{A,B\} &= AB + BA \quad\text{ (Antikommutator)}
\end{align*}
Wir sagen, dass \(A\) und \(B\) kommutieren \(\Leftrightarrow [A,B] = \mathbb{0}_n \Leftrightarrow AB = BA\).\\
\(A\) und \(B\) antikommutieren \(\Leftrightarrow \{A,B\} = \mathbb{0}_n\)
}

\thm{}{
Sei \(A \in M^n(\KK)\), sodass \(A\) mit jedem \(B \in M^n(\KK)\) kommutiert. Dann ist \(A = \alpha \cdot \mathbb{1}_n\) für gewisses \(\alpha \in \KK\).
}

\dfn{Spur einer quadratischen Matrix}{
\[\mathrm{Sp}\,A = \mathrm{sp}\,A = \trace{A} = \mathrm{tr}\,A = \sum_{i=1}^{n} a_{ii}\]
}

\thm{}{
\(A \in M^{m,n}, B \in M^{n,m}\)\\
Dann ist \(AB \in M^m, BA \in M^n\) und \(\trace(AB) = \trace(BA)\).
}

\thm{}{
Für \(\sigma,\tau \in S_n\) gilt immer
\[\veps(\sigma\tau) = \veps(\sigma) \cdot \veps(\tau)\]
}

\section{\texorpdfstring{\(\RR^n\) bzw. \(\CC^n\) als Raum der Spaltenvektoren}{Rⁿ bzw. Cⁿ als Raum der Spaltenvektoren}}

\section{Permutationen}

\thm{}{
Für \(\sigma, \tau \in S_n\) gilt immer
\[\veps(\sigma \tau) = \veps(\sigma) \cdot \veps(\tau) \,.\]
}

\section{Determinanten}

\dfn{}{
\[\det A = \sum_{\sigma \in S_n} \veps(\sigma) \cdot (a)_{\sigma} = \sum_K \veps(K)a_{1k_1} \cdot \hdots \cdot a_{nk_n}\]
}

\thm{Entwicklungssatz (\textsc{Laplace})}{
\(A = \Le(a_{ij}\Ri)_{i=1,\ldots,n}^{j=1,\ldots,n}\)\\
\(\begin{pmatrix}
	a_{11} & \dots & a_{1k} & \dots & a_{1n}\\
	\vdots &        & \vdots &        & \vdots\\
	a_{\ell 1} & \dots & a_{\ell k} & \dots & a_{\ell n}\\
	\vdots &        & \vdots &        & \vdots\\
	a_{n1} & \dots & a_{nk} & \dots & a_{nn}
\end{pmatrix}\)
\begin{align*}
	\det A &= \sum_{j=1}^{n}(-1)^{\ell+j} \cdot a_{\ell j} \cdot M_{\ell j}\\
	&= \sum_{i=1}^{n}(-1)^{i+k} \cdot a_{ik} \cdot M_{ik}
\end{align*}
}

\section{Inverse Matrizen}

\dfn{}{
Sei \(A \in M^{m,n}\).\\
Man nennt \(B_{\text{L}} \in M^{n,m}\) linksinvers zu \(A\)
\[\Leftrightarrow \quad B_{\text{L}} \cdot A = \mathbb{1}_n \in M^{n,n}.\]
Man nennt \(B_{\text{R}} \in M^{n,m}\) rechtsinvers zu \(A\)
\[\Leftrightarrow \quad A \cdot B_{\text{R}} = \mathbb{1}_m \in M^{m,m}.\]
}
\thm{}{
	Sei \(A \in M^n\), also \(m=n\). Dann sind folgende Aussagen äquivalent:
\begin{enumerate}[label=(\arabic*)]
	\item \(A\) besitzt eine linksinverse Matrix \(B_{\text{L}}\).
	\item \(A\) besitzt eine rechtsinverse Matrix \(B_{\text{R}}\).
	\item \(A\) besitzt eine inverse Matrix \(A^{-1}\).
	\item \(\det{A} \neq 0\).
\end{enumerate}
}

\dfn{}{
Man nennt \(A \in M^n\)
\begin{itemize}
	\item regulär, falls \(\det{A} \neq 0\)
	\item singulär, falls \(\det{A} = 0\)
\end{itemize}
\(A \in M^n\) invertierbar \(\Leftrightarrow\) regulär.
}

\thm{}{
Sei \(A \in M^n\) regulär.\\
Dann besitzt (*) für jede beliebige rechte Seite genau eine Lösung.
\[\mathbb{x} =
\begin{pmatrix}
	x_1\\\vdots\\x_n
\end{pmatrix} = A^{-1} \cdot \mathbb{f} = A^{-1} \cdot
\begin{pmatrix}
	f_1\\\vdots\\f_n
\end{pmatrix}\]
}

\section{Der Rang einer Matrix}

\dfn{}{
\(A \in M^{m,n}\) besitzt den Rang \(r = r(A) \geqslant 1\), falls es eine Minor \(\tilde{A}\) der Ordnung \(r\) gibt, mit \(\det{\tilde{A}} \neq 0\), und falls für alle Minoren der Ordnungen \(> r\) deren Determinanten gleich null sind.
}

\dfn{}{
Ein System von Spalten(vektoren) \(\mathbb{x},\hdots,\mathbb{x}_k\) nennt man linear unabhängig, falls
\[\alpha_1 \mathbb{x}_1 + \hdots + \alpha_k \mathbb{x}_k = \mathbb{0}_n \enspace\Leftrightarrow\enspace \alpha_1 = \hdots = \alpha_k = 0\]
Analog: Zeilen(vektoren)\\
Sonst: linear abhängig
}

\dfn{}{
Der Spaltenrang \(r_{\text{s}}(A)\) von \(A \in M^{m,n}\) ist die größtmögliche Anzahl linear unabhängiger Spalten von \(A\).
}

\dfn{}{
Der Zeilenrang \(r_{\text{z}}(A)\) von \(A \in M^{m,n}\) ist die größtmögliche Anzahl linear unabhängiger Zeilen von \(A\).
}

\thm{Satz vom Rang}{\[r(A) = r_{\text{z}}(A) = r_{\text{s}}(A)\]}

\dfn{}{Die Dimension eines Vektorraums ist die größtmögliche Anzahl linear unabhängiger Vektoren aus diesem Raum.}

\dfn{Lineare Hülle}{
\[\Le\{\alpha_1 f_1 + \hdots + \alpha_k f_k \colon \alpha_1,\hdots,\alpha_k \in \KK\Ri\} \enspace=\enspace V\{f_1,\hdots,f_k\} \enspace=\enspace \Vektor_{j=1}^k \{f_x\}\]
lineare Hülle des Systems \(\{f_1,\hdots,f_k\}\)
}

\thm{}{
\(\{f_1,\hdots,f_x\} \sbst E\) linear unabhängig\vspace{4pt}\\
\(k \in \NN, \Vektor_{j=1}^k f_j = E\)
\[\implies \quad \dim{E} = k\]
}

\thm{}{
\(\dim{E} = k\in\NN\)\\
\(\{f_1,\hdots,f_k\}\) linear unabhängig\\
\[\implies \quad E = V\{f_1,\hdots,f_k\}\]
}

\dfn{}{
\(\{f_1,\hdots,f_k\}\) ist eine Basis in \(E\)
\[\Leftrightarrow \quad
\begin{cases}
	\{f_1,\hdots,f_k\} \text{ linear unabhängig }\\
	\Vektor_{j=1}^k\{f_j\} = E \text{ vollständig }
\end{cases}\]
}

\dfn{}{Man nennt \(L \sbst E\) einen Unterraum von \(E\), falls
\[\begin{rcases}
	\forall f,g \in L\\
	\forall \alpha, \beta \in \KK
\end{rcases} \implies \alpha f + \beta g \in L\]
}


\thm{Dimensionssatz}{
\(A \in M^{m,n}\)\\
\[\dim{\core(A)} + \dim{W(A)} = \dim{\core(A)} + r(A) = n\]
}

\thm{}{
\(A \cdot \mathbb{x} = \mathbb{f}\) ist für jedes \(\mathbb{f} \in \KK^n\) lösbar, genau dann, wenn
\[r(A) = m \,.\]
}

\thm{}{
Das LGS \(A\mathbb{x} = \mathbb{f}\) besitze eine Lösung \(\mathbb{x} = \mathbb{x}_p \in \KK[n]\). Dann ist diese eindeutig genau dann, wenn
\[\ker(A) = \Le\{\mathbb{0}_{\KK[n]}\Ri\} \,.\]
}

\section{LGS: Allgemeiner Fall}

\section{Das Spektrum. Eigenvektoren. Resolvente.}

\dfn{Charakteristisches Polynom}{
\[d_A(\lambda) = \det(A - \lambda\id)\]
}

\dfn{}{
\(\mu_1, \hdots, \mu_k\) Eigenwerte von \(A\)\\
\(\tau_1, \hdots, \tau_k\) Algebraische Vielfachheit
}

\dfn{}{
\(\sigma(A) = \{\mu_1, \hdots, \mu_k\} \sbst \CC\) Spektrum von \(A\).
}

\dfn{}{
\(\varkappa = \dim{E_{\mu}}\) geometrische Vielfachheit des Eigenwerts \(\mu\).
}

\dfn{}{
\(\rho(A) = \CC \backslash \sigma(A)\) Resolventenmenge.
\begin{align*}
	\mu \in \rho(A) &\Leftrightarrow d_A(\mu) = \det(A - \mu\id) \neq 0\\
	&\Leftrightarrow A - \mu\id \text{ invertierbar}\\
	&\Leftrightarrow \text{Es existiert } (A - \mu\id)^{-1}
\end{align*}
}

\dfn{}{
\(\mu \in \rho(A)\)\\
\[\Gamma_{\mu}(A) = (A - \mu\id)^{-1}\]
Resolvente von \(A\) im Punkt \(\mu \in \rho(A)\).
\[\Gamma_{\mu}(A) \colon \rho(A) = \CC \backslash \sigma(A) \to M^{n,n}\]
}

\dfn{}{
Für \(A \in M^s\) und \(p(z) = c_n z^n  + \hdots + c_1 z + c_0 \qquad c_k,z \in \CC\)\\
sei\\
\[p(A) = c_n A^n  + \hdots + c_1 A + c_0 \id\]
\[p \colon M^s \to M^s\]
}


\dfn{}{
Eine Matrix \(B \in M^{n,n}\) heißt diagonalisierbar, genau dann, wenn sie zu einer Diagonalmatrix ähnlich ist. Das heißt, es gibt \( A = \diag{a_1, \hdots, a_n}\) und es gibt \(X \in M^{n,n}, \det{X} \neq 0\), sodass
\[B = X^{-1}AX\]
}

\dfn{}{
\(f \colon \sigma(A) \to \CC\)
\[f(A) \coloneq \diag{f(\lambda_1), \hdots, f(\lambda_n)}\]
}

\dfn{}{
\(f \colon \sigma(B) \to \CC\)
\begin{align*}
	B &= X^{-1} \cdot \diag{\lambda_1, \hdots, \lambda_n}X\\
	f(B) &= X^{-1} \cdot \diag{f(\lambda_1), \hdots, f(\lambda_n)}X
\end{align*}
}

\section{Ähnlichkeit von Matrizen}

\section{Orthogonale und unitäre Matrizen}

\dfn{}{\[\mathbb{x} \perp\ \mathbb{y} \Leftrightarrow \langle x, y \rangle = 0 \text{ orthogonal}\]}

\dfn{}{Ein Vektor \(\mathbb{x} \in \KK\) heißt normiert, falls \(\norm{\mathbb{x}} = 1\).}

\dfn{}{
Ein System von Vektoren \(\{\mathbb{y}_1, \hdots, \mathbb{y}_k\}\) heißt orthonormiert (ON), genau dann, wenn
\[\mleft\langle \mathbb{y}_j, \mathbb{y}_\ell \mright\rangle = \updelta_{j\ell} \qquad j,\ell = 1, \hdots, k\]
ONS System von orthonormalen Vektoren.\\
Bildet ein ONS eine Basis im \(\KK^n\), spricht man von einer Orthonormalbasis (ONB).
}

\section{Symmetrische und Hermitesche Matrizen}

\dfn{}{
\(\KK = \RR\)\\
Wir nennen \(A \in M^n\) symmetrisch, genau dann, wenn \(A = A^{\mathsf{T}}\).
}

\dfn{}{
\(\KK = \CC\)\\
Wir nennen \(A \in M^n\) hermitsch (bzw. selbstadjungiert), genau dann, wenn \(A = A^*\).
}

\thm{}{
Sei \(A = A^{\mathsf{T}}\) (falls \(\KK = \RR\)) bzw. \(A = A^*\) (falls \(\KK = \CC\)). Es seien \(\lambda_1, \lambda_2\) Eigenwerte von \(A\), und \(\mathbb{x}_1, \mathbb{x}_2\) zugehörige Eigenvektoren. Aus \(\lambda_1 \neq \lambda_2\) folgt dann
\[\langle\mathbb{x}_1, \mathbb{x}_2\rangle = 0, \text{ d.h. }\mathbb{x}_1 \perp \mathbb{x}_2.\]
}

\dfn{}{
\(\KK = \RR\)\\
Man nennt \(A \in M^n(\RR)\) orthogonal diagonalisierbar, genau dann, wenn eine orthogonale Matrix \(Y \in M^n(\RR)\) existiert, sodass
\[Y^{-1}AY = \mathrm{diag}\Le\{\lambda_1, \hdots, \lambda_2\Ri\}.\]
Mit \(Y^{-1} = Y^{\mathsf{T}}\)
}

\dfn{}{
\(\KK = \CC\)\\
Man nennt \(A \in M^n(\CC)\) unitär diagonalisierbar, genau dann, wenn eine unitäre Matrix \(Y \in M^n(\CC)\) existiert, sodass
\[Y^{-1}AY = \mathrm{diag}\Le\{\lambda_1, \hdots, \lambda_2\Ri\}.\]
Mit \(Y^{-1} = Y^*\)
}

\thm{}{
\(\KK = \RR\)\\
Jede symmetrische Matrix \(A \in M^n(\RR)\) besitzt eine ONB aus Eigenwerten im \(\RR^n\) und ist somit orthogonal diagonalisierbar.\\
\(\KK = \CC\)\\
Jede hermitsche Matrix \(A \in M^n(\CC)\) besitzt eine ONB aus Eigenwerten im \(\CC^n\) und ist somit unitär diagonalisierbar.
}

\thm{Spektralsatz für unitäre Matrizen}{
Unitäre Matrizen sind unitär diagonalisierbar.
}

\section{Wechsel des Koordinatensystems – Basiswechsel}

\thm{}{
\(A \in M^n\) ist unitär diagonalisierbar, falls
\[AA^* = A^*A.\]
Solche Matrizen nennt man normal.
}

\dfn{}{
\[A \in M^{n,n}(\CC) \text{ normal } \Leftrightarrow AA^* = A^*A.\]
}

\thm{}{
\(A \in M^n\) ist genau dann unitär diagonalisierbar, wenn \(A\) normal ist.
}

%Anmerkung: Weirde Wiederholungen

\thm{}{
Zwei normale Matrizen \(A\) und \(B\) kommutieren genau dann, wenn sie eine gemeinsame ONB von Eigenvektoren besitzen.
}

\section{Direkte und orthogonale Summen von Unterräumen}

\dfn{}{
\[F + G \coloneq \Le\{h \in V \colon h = f + g, \enspace f \in F, \enspace g \in G\Ri\}\]
\(F + G\) ist ein Unterraum:
\[h_1 = f_1 + g_1, \enspace h_2 = f_2 + g_2\]
\[\implies \quad h = \alpha h_1 + \beta h_2 = {\underbrace{\Le(\alpha f_1 + \beta f_2\Ri)}_{f \in F}} + {\underbrace{\Le(\alpha g_1 + \beta g_2\Ri)}_{g \in G}}\]
}

\dfn{}{
Wir nennen \(H = F + G\) eine direkte Summe von Unterräumen, wenn für jedes \(h \in H\) die Darstellung \(h = f + g, \enspace f \in F, \enspace g \in G\) eindeutig ist:
\[H = F \mathrel{\dot{+}} G\]
}

\thm{}{
\(F, G \sbst V\) Unterräume\\
\(H = F + G\) (im Allgemeinen nicht direkt)
\[\implies \quad \dim(F + G) + \dim(F \cap G) = \dim F + \dim G\]
}

\dfn{}{
\[H = F_1 + \hdots + F_m = \Le\{h \in V \colon h = f_1 + \hdots + f_m, \enspace f_j \in F_j, \enspace j = 1,\hdots,m\Ri\}\]
Falls die Zerlegung \(h = f_1 + \hdots + f_m\) für jedes \(h \in H\) eindeutig ist, dann nennt man die Summe direkt.
\[H = F_1 \mathrel{\dot{+}} F_2 \mathrel{\dot{+}} \hdots \mathrel{\dot{+}} F_m\]
}

\dfn{}{
\(F \perp G \Leftrightarrow f \perp g \quad \forall f \in F, g \in G\)\\
Sind \(F, G\) Unterräume von \(V\) und \(F \perp G\)
\[H = F + G = F \oplus G = F \mathrel{\dot{+}} G\]
}

\thm{}{
Jede Matrix \(A \in M^n\) mit einem Eigenwert \(\lambda\) der algebraischen Vielfachheit \(\tau = n\) und der geometrischen Vielfachheit \(\varkappa = 1\) lässt sich in der Form
\[A = X^{-1}J^{(n)}(\lambda)X\]
mithilfe einer regulären Matrix \(X \in M^n\) darstellen.
}

\section{Orthogonale Projektionen}

\dfn{}{
\(\xi_k = \Le\langle x, f_k\Ri\rangle, \enspace k = 1,\hdots,n\) nennt man die \textsc{Fourier}-Koeffizienten von \(x\) bzgl. der ONB \(\mathbb{f}\).
}

\thm{}{
\(x \in V\)\\
\(\xi_j = \Le\langle x, f_j \Ri\rangle, \enspace j = 1,\hdots,s\)\\
Es gilt immer:
\[\norm{x - \sum_{j=1}^{s} \beta_j f_j} \geqslant \norm{x - \sum_{j=1}^{s} \xi_j f_j}\]
Gleichheit (\enquote{\(=\)}) \(\Leftrightarrow \beta_j = \xi_j, \enspace j=1,\hdots,s\)
}

\thm{}{
Für jedes \(y = \beta_1 f_1 + \hdots + \beta_s f_s \in F\) gilt
\[\norm{x - y} \geqslant \norm{x - x_F}\]
mit \enquote{\(=\)} \(\Leftrightarrow y = x_F\).
}

\thm{Projektionssatz}{
Sei \(F \sbst V\) ein Unterraum. Dann existiert für jedes \(x \in V\) genaue eine Zerlegung \(x = x_{F} + x_G\) mit \(x_F \in F\) und \(x_G \perp F\).
}

\dfn{}{
\(P_F \colon V \to F\) mit \(P_F \,x = x_F\) nennt man die orthogonale Projektion auf \(F\).
}

\dfn{}{
\(G = \{g \in V \colon g \perp F\} = F^{\perp}\) nennt man das orthogonale Komplement zu \(F\).
}

\section{Selbstadjungierte Operatoren und quadratische Formen}

\thm{}{
Zu jeder sesqui-linearen Form \(a\) existiert genau eine lineare Abbildung \(\mathcal{A} \colon V \to V\), so, dass
\[a[x,y] = \langle \mathcal{A}x, y \rangle\]
mit \(\mathcal{A} \colon V \to V \) linear, gilt.
} %Der hier kommt doch safe nicht dran

\dfn{}{
Der zu \(\mathcal{A}\) adjungierte Operator \(A^*\) ist durch \(a^*\) gegeben:
\[\Le\langle A^*x, y\Ri\rangle = a^*[x,y] = \overline{a[y,x]}.\]
Mit \(\Le\langle A^*x, y\Ri\rangle = \langle x, \mathcal{A}y\rangle\) und \(\overline{a[y,x]} = \overline{\langle Ay, x\rangle}\):
\[\langle x, \mathcal{A}y\rangle = \overline{\langle Ay, x\rangle}.\]
}

\section{Stetige lineare Operatoren}

\(E, F\) Vektorräume über \(\KK \in \{\RR, \CC\}\) mit Normen \(\norm{\cdot}_E\) und \(\norm{\cdot}_F\).\\
\(F \colon E \to F\) linear.\\

\dfn{}{
Ein linearer Operator \(T \colon E \to F\) heißt beschränkt, genau dann, wenn
\[\exists_{C>0} \enspace \forall_{x \in E} \colon \enspace \norm{Tx}_F \leqslant C \cdot \norm{x}_E \,.\]
}

\thm{}{
Folgende Aussagen sind äquivalent:
\begin{enumerate}[label=(\arabic*)]
	\item \(T\) ist von \(E\) nach \(F\) beschränkt.
	\item \(T\) ist in \(0_E\) stetig.
	\item \(T\) ist auf \(E\) stetig.
\end{enumerate}
}

\dfn{}{
\(\mathcal{L}(E,F)\) sei die Menge aller stetigen linearen Abbildungen \(T \colon E \to F\).
}

\thm{Operatornorm}{
\(\mathcal{L}(E,F)\) ist ein Vektorraum über \(\KK\).
\[\norm*{T}_{\mathcal{L}(E,F)} = \sup_{x \in E, x \neq 0} \frac{\norm{Tx}_F}{\norm{x}_E}\] %\norm*{} setzt nicht skalierte Delimiter, da skalierte Delimiter zu schlechtem Spacing rechts führen. (Mathopen und Mathclose Symbole)
ist eine Norm auf \(\mathcal{L}(E,F)\). Sind \(E\) und \(F\) vollständig, dann ist auch \(\mathcal{L}(E,F)\) vollständig.
}

\chapter{Zur Diff.-rechnung für Funktionen mehrerer Var.}

\section{Differenzierbarkeit}

\dfn{Partielle Ableitung in \(x_j\)-Richtung}{
	\begin{multline*}
		\eval{\pdv{f}{x_j}}_{x=x^{(0)}} = \eval{\odv{\Le(f\Le(x^{(0)} + t \mathbb{e}_j\Ri)\Ri)}{t}}_{t=0} \\[5pt]
		= \lim_{\tau \to 0} \frac{f\Le(x_1^{(0)}, \hdots, x_{j-1}^{(0)}, x_j^{(0)} + \tau, x_{j+1}^{(0)}, \hdots, x_n^{(0)}\Ri) - f\Le(x_1^{(0)}, \hdots, x_n^{(0)}\Ri)}{\tau}
	\end{multline*}
%\[\eval{\pdv{f}{x_j}}_{x=x^{(0)}} = \eval{\odv{\Le(f\Le(x^{(0)} + t %\mathbb{e}_j\Ri)\Ri)}{t}}_{t=0} = \lim_{\tau \to 0} \frac{f\Le(x_1^{(0)}, \hdots, %x_{j-1}^{(0)}, x_j^{(0)} + \tau, x_{j+1}^{(0)}, \hdots, x_n^{(0)}\Ri) - f\Le(x_1^{(0)}, %\hdots, x_n^{(0)}\Ri)}{\tau}\]
}

\dfn{}{
\(f\) ist im Punkt \(x^{(0)} \in U\) (\textsc{Fréchet})-differenzierbar, genau dann, wenn \(A \in \mathcal{L}(E,F)\) mit
\[f\Le(x^{(0)} + h\Ri) = f\Le(x^{(0)}\Ri) + A \cdot h + \mathcal{o}(h), \quad h \to 0_E \,.\]
Dann ist \(\eval{\d f}_{x=x^{(0)}} = \mathcal{A}\).
}

\dfn{}{
Die Richtungsableitung \(\mathrm{D}f\Le(x^{(0)}\Ri)h\) ist gegeben als
\[\mathrm{D}f\Le(x^{(0)}\Ri)h \enspace=\enspace \lim_{t \to 0} \frac{f\Le(x^{(0)} + th\Ri) - f\Le(x^{(0)}\Ri)}{t} \enspace=\enspace \eval{\odv{}{t} f\Le(x^{(0)} + th\Ri)}_{t=0}\,.\]
(jeweils für fixiertes \(h\))
}

\thm{}{
Wenn \(f \colon U \sbst E \to F\) in \(x^{(0)} \in U\) \textsc{Fréchet}-differenzierbar ist, dann existieren alle Richtungsableitungen \(\mathrm{D}f\Le(x^{(0)}\Ri)h\) (für alle \(h \in E\)).
}

\dfn{}{
\(f\) ist in \(x^{(0)} \in U\) schwach differenzierbar (\textsc{Gâteaux}-differenzierbar), genau dann, wenn
\begin{enumerate}[label=(\arabic*)]
	\item \(\mathrm{D}f\Le(x^{(0)}\Ri)h\) existiert für alle \(h \in E\).
	\item \(\mathrm{D}f\Le(x^{(0)}\Ri)h\) ist linear in \(h\).
	\item \(\mathrm{D}f\Le(x^{(0)}\Ri)h\) ist stetig in \(h\).
\end{enumerate}
\[f_{\text{s}}'\Le(x^{(0)}\Ri)h = \mathrm{D}f\Le(x^{(0)}\Ri)h; \quad f_{\text{s}}'\Le(x^{(0)}\Ri) \in \mathcal{L}(E,F)\]
}

\thm{}{
Wenn \(f\) in \(x^{(0)}\) \textsc{Fréchet}-differenzierbar ist, dann ist \(f\) in \(x^{(0)}\) schwach differenzierbar und \(f_{\text{s}}'\Le(x^{(0)}\Ri) = f'\Le(x^{(0)}\Ri)\).
}

\thm{}{
\begin{align*}
	x &\mapsto \pdv{f}{x_j}(x), \quad j=1,\hdots,n \text{ stetig in } x^{(0)} \quad\implies\quad \exists f_{\text{s}}'\Le(x^{(0)}\Ri)\\
	x &\mapsto \pdv{f}{x_j}(x), \quad j=1,\hdots,n \text{ stetig in } B_{\veps}\Le(x^{(0)}\Ri) \quad\implies\quad \exists f'\Le(x^{(0)}\Ri)
\end{align*}
}

\section{Produkt- und Kettenregel}

\(\Le(E, \norm{\cdot}_E\Ri)\)\vspace{3pt}\\
\(\Le(F, \norm{\cdot}_F\Ri)\)\vspace{3pt}\\
\(U \sbst E\) offen, \(x \in U\)

\thm{}{
\(f \colon U \to F\) und \(\alpha \colon U \to \RR\) seien in \(x^{(0)} \in U\) \textsc{Fréchet}-differenzierbar. Dann ist \(\alpha \cdot f \colon U \to F\) in \(x^{(0)}\) \textsc{Fréchet}-differenzierbar.
\[(\alpha f)'\Le(x^{(0)}\Ri) = \alpha \Le(x^{(0)}\Ri) f'\Le(x^{(0)}\Ri) + f\Le(x^{(0)}\Ri) \alpha '\Le(x^{(0)}\Ri)\]
}

\thm{}{
\(f\) sei in \(x^{(0)} \in U\) \textsc{Fréchet}-differenzierbar.\\
\(g\) sei in \(y^{(0)} \in V\) \textsc{Fréchet}-differenzierbar.
\[\implies \quad \text{Dann ist } g \circ f \colon U \to G \text{ \textsc{Fréchet}-differenzierbar.}\]

\[\Le(g \circ f\Ri)'\Le(x^{(0)}\Ri) = g'\Le(f\Le(x^{(0)}\Ri)\Ri) \cdot f'\Le(x^{(0)}\Ri)\]
}


\section{Hauptsatz der Differenzialrechnung}

\thm{}{
\(f\) sei auf \(\overline{ab}\) \textsc{Gâteaux}-differenzierbar.\\
Dann gilt:
\begin{enumerate}[label=(\arabic*)]
	\item \(\displaystyle \norm{f(b) - f(a)}_F \enspace\leqslant\enspace \sup_{x \in \overline{ab}} \Le(\norm{f_{\text{s}}'(a)}_{\mathcal{L}(E,F)} \cdot \norm{b-a}_E\Ri)\)
	\item \(\displaystyle \norm{f(b) - f(a) - f_{\text{s}}'(a)(b-a)}_F \enspace\leqslant\enspace \sup_{x \in \overline{ab}} \Le(\norm{f_{\text{s}}'(x) - f_{\text{s}}'(a)}_{\mathcal{L}(E,F)} \cdot \norm{b-a}\Ri)\)
\end{enumerate}
}

\section{Ableitungen höherer Ordnung}

\dfn{}{
Ist \(g\) in \(x^{(0)} \in U\) partiell in \(x_k\) differenzierbar, dann sei
\[\pdv{g\Le(x^{(0)}\Ri)}{x_k} = \eval{\Le(\pdv{}{x_k}\Le(\pdv{f}{x_j}\Ri)\Ri)}_{x=x^{(0)}} = \eval{\pdv{f}{x_k,x_j}}_{x=x^{(0)}}.\]
}

\thm{Symmetriesatz}{
Wenn in \(U\) beide partiellen Ableitungen
\[\pdv{f}{x_k,x_j} \text{ und } \pdv{f}{x_j,x_k}\]
existieren, und beide stetig sind, dann gilt auf \(U\)
\[\pdv{f}{x_k,x_j} = \pdv{f}{x_j,x_k} \,.\]
}

\dfn{}{
Wenn \(f'(\cdot)\) in \(x^{(0)} \in U\) \textsc{Fréchet}-differenzierbar ist, dann sei
\[f''\Le(x^{(0)}\Ri) = \eval{\Le(f'(\cdot)\Ri)'}_{x=x^{(0)}} \in \mathcal{L}\Le(E,\mathcal{L}(E,F)\Ri)\]
}

\section{\texorpdfstring{Der Satz von \textsc{Taylor}}{Der Satz von Taylor}}

\(f \colon U \sbst E \to F\), \(U\) offen\\
\(f\) ist in \(U\) \(m\)-fach \textsc{Fréchet}-differenzierbar

\thm{Die Formel von \textsc{Taylor}}{
\[f\Le(x^{(0)} + h\Ri) = f\Le(x^{(0)}\Ri) + \sum_{k=1}^{m} \frac{1}{k!} f^{(k)}\Le(x^{(0)}\Ri) h^k + r_m\Le(x^{(0)},h\Ri)\]
wobei:
\[r_m\Le(x^{(0)}, h\Ri) = \mathcal{o}\Le(\norm{h}_E^m\Ri) \qquad \text{ für } h \to 0_E \,.\]
}

\section{Extremwerte von Funktionen mit mehreren Veränderlichen}

\(\Le(E, \norm{\cdot}_E\Ri)\)\vspace{3pt}\\
\(F = \RR\)\\
\(U \sbst E\) offen\\
\(f \colon U \sbst E \to \RR\)

\dfn{}{
\(f\) nimmt in \(x^{(0)}\) ein\\
lokales Maximum an
\[\Leftrightarrow \quad \exists_{\veps > 0} \enspace \forall_{x \in U, \norm{x-x^{(0)}} < \veps} \colon \enspace f(x) \leqslant f\Le(x^{(0)}\Ri)\]
lokales Maximum an
\[\Leftrightarrow \quad \exists_{\veps > 0} \enspace \forall_{x \in U, \norm{x-x^{(0)}} < \veps} \colon \enspace f(x) \geqslant f\Le(x^{(0)}\Ri)\]
}

\thm{Notwendiges Kriterium}{
\(f \colon U \sbst E \to \RR\) nehme in den inneren Punkten \(x^{(0)} \in U\) einen lokalen Extremwert an. Wenn die Richtungsableitung \(\mathrm{D}f\Le(x^{(0)}\Ri)h\) existiert, so muss dann
\[\mathrm{D}f\Le(x^{(0)}\Ri)h = 0\]
gelten.
}

\section{Der Satz über implizite Funktionen}

\dfn{Lokale Auflösbarkeit}{
\(\Le(x^{(0)}, y^{(0)}\Ri) \in W\)\vspace{3pt}\\
\(\Phi\Le(x^{(0)}, y^{(0)}\Ri) = 0\)\\
\(\Phi(x,y) = 0\) ist lokal in einer Umgebung von \(\Le(x^{(0)}, y^{(0)}\Ri)\) zu \(y = f(x)\) auflösbar, genau dann, wenn
\[\exists_{\veps > 0} \enspace \exists_{\delta > 0} \enspace \exists f \colon \enspace U_{\veps}\Le(x^{(0)}\Ri) \to U_{\delta}\Le(y^{(0)}\Ri)\]
\(U_{\veps}\Le(x^{(0)}\Ri) \times U_{\delta}\Le(y^{(0)}\Ri) \sbst W\)
\begin{enumerate}[label=(\arabic*)]
	\item \(\displaystyle \forall_{x \in U_{\veps}\Le(x^{(0)}\Ri)} \colon \enspace \Phi(x,f(x)) = 0\)
	\item \(\displaystyle \forall_{(x,y) \in U_{\veps}\Le(x^{(0)}\Ri) \times U_{\delta}\Le(y^{(0)}\Ri)} \colon \enspace \Phi(x,y) = 0 \quad \implies \quad  y = f(x)\)
\end{enumerate}
}

\thm{zu impliziten Funktionen}{
\(W \sbst \RR_x^m \times \RR_y^n\) offen\vspace{3pt}\\
\(\Le(x^{(0)}, y^{(0)}\Ri) \in W\)\vspace{3pt}\\
\(\Phi \colon W \to \RR^n\)\vspace{3pt}\\
\(\Phi\Le(x^{(0)}, y^{(0)}\Ri) = 0\)\vspace{3pt}\\
\(\Phi\) sei aus der Klasse \(C^p, \enspace p \geqslant 1\) (d.h. alle partiellen Ableitungen bis zur Ordnung \(p\) existieren und sind stetig).\vspace{3pt}\\
\(\Phi_y'\Le(x^{(0)}, y^{(0)}\Ri)\) ist invertierbar
\[\implies \quad \Phi \text{ ist lokal zu } y = f(x) \text{ auflösbar und } f \text{ ist von der Klasse } C^p.\]
\[f'\Le(x^{(0)}\Ri) = - \Le[\Phi_y'\Le(x^{(0)}, y^{(0)}\Ri)\Ri]^{-1} \Phi_x'\Le(x^{(0)}, y^{(0)}\Ri)\]
}

\section{Umkehrfunktion}

\dfn{}{
\(f \colon U \to V\) ist ein \(C^p\)-Diffeomorphismus genau dann, wenn
\begin{itemize}
	\item \(f \colon U \to V\) bijektiv
	\item \(f, f^{-1}\) aus der Klasse \(C^p\)
\end{itemize}
}

\thm{}{
\(f \colon U \sbst \RR[m] \to \RR[m], \enspace U\) offen\\
\(f \in C^p\)\\
\(p \geqslant 1, \enspace p \in \NN\)\vspace{3pt}\\
\(y^{(0)} = f\Le(x^{(0)}\Ri) \text{ für } x^{(0)} \in U\)\vspace{3pt}\\
Sei \(f'\Le(x^{(0)}\Ri)\) invertierbar.
\[\implies \quad \exists \text{ offene Mengen } U_{x^{(0)}}, \enspace V_{y^{(0)}} \neq 0\]
\(x^{(0)} \in U_{x^{(0)}}, \enspace y^{(0)} \in V_{y^{(0)}}; \enspace f \colon U_{x^{(0)}} \to V_{x^{(0)}}\)\\
\(+ \quad \Le(f^{-1}\Ri)_y' = - \Le[\Phi_x'\Ri]^{-1} \cdot \Phi_y' = \Le[f_x'\Ri]^{-1}\)
}


\section{\texorpdfstring{Darstellung von Gradient und \textsc{Laplace} in verschiedenen Koordinatensystemen}{Darstellung von Gradient und Laplace in verschiedenen Koordinatensystemen}}

\section{Extremwerte unter Nebenbedingungen}

\dfn{\textsc{Lagrange}-Funktion}{
\[\mathscr{L}(x,y;\lambda) = f(x,y) - \lambda \cdot F(x,y)\]
\(\lambda:\) \textsc{Lagrange}-Faktor
}

\dfn{}{
\begin{enumerate}[label=(\arabic*)]
	\item \(F\Le(\tilde{x}^{(0)}\Ri) = \mathbb{0}_n\)
	\item \(\displaystyle\exists_{\veps > 0} \enspace \forall_{\tilde{x} \in U_{\veps}\Le(\tilde{x}^{(0)}\Ri), \, F \Le(\tilde{x}\Ri) = \mathbb{0}_n} \colon  \enspace f\Le(\tilde{x}^{(0)}\Ri) \geqslant f\Le(\tilde{x}\Ri) \text{ bzw. } f\Le(\tilde{x}^{(0)}\Ri) \leqslant f\Le(\tilde{x}\Ri)\)
\end{enumerate} %Die Klammern von F(x) und F\Le(x\Ri) sind gleich groß, doch der Abstand zwischen linker Klammer und F ist unterschiedlich. Wieso?
}

\chapter{Funktionenfolgen}

\section{Doppelfolgen. Gleichmäßigkeit.}

\((M,d)\) metrischer Raum

\dfn{Doppelfolge}{
\(a \colon \NN \times \NN \to M\)\\
also:
\[ \Le(a_{m,n}\Ri)_{m \in \NN, \, n \in \NN} = \Le(a_{m,n}\Ri)\]
}

\dfn{}{
\(A(\cdot)\) Aussageform, Eigenschaft\\
\(X\) Variablenmenge\\
\(A\) ist für alle \(x \in X\) punktweise erfüllt, genau dann, wenn
\[\forall_{x \in X} \colon  A(x) \text{ wahr }\]
Dabei können die Parameter, die in \(A\) eingehen von \(x\) abhängen.
}

\dfn{}{
\(A\) ist gleichmäßig bzgl. \(x \in X\) erfüllt, genau dann, wenn \(\forall_{x \in X} \colon A(x)\) wahr ist, und wenn die Parameter, welche in \(A\) eingehen, unabhängig von \(x\) gewählt werden können.
}

\thm{Satz über das Vertauschen von Grenzwerten}{
\((M,d)\) vollständiger metrischer Raum\\
\(a \colon \NN \times \NN \to M\)
\begin{enumerate}[label=(\arabic*)]
	\item \(\displaystyle\forall_{n\in\NN} \enspace \exists \lim_{m \to \infty} a_{m,n} = v_n\)
	\item \(\displaystyle\forall_{m\in\NN} \enspace \exists \lim_{n \to \infty} a_{m,n} = u_m\)
\end{enumerate}
Einer dieser beiden Grenzwerte werde gleichmäßig angenommen.\\
\(\implies\)
\begin{itemize}
	\item \(\Le(u_m\Ri)\) und \(\Le(v_n\Ri)\) konvergieren
	\item \(\displaystyle \lim_{m \to \infty} u_m = \lim_{n \to \infty} v_n\)
\end{itemize}
Damit darf ich unter der zusätzlichen Annahme der Gleichmäßigkeit eines der Grenzwerte die doppelten Grenzwerte vertauschen:
\[\lim_{m \to \infty} \Le(\lim_{n \to \infty} a_{m,n}\Ri) = \lim_{n \to \infty} \Le(\lim_{m \to \infty} a_{m,n}\Ri)\]
}

\section{Funktionenfolgen}

\((M,d) = (\RR, \abs{\cdot}), \enspace I \sbst \RR, \enspace I \neq \emptyset\)\\
\(f, f_n \colon I \to \RR, \enspace n \in \NN\)\\
\(\Le(f_n\Ri)_{n \in \NN}\) Funktionenfolge

\dfn{}{
\(f_n \overset{n \to \infty}{\longrightarrow} f\) punktweise bzgl. \(x \in I\) genau dann, wenn
\[\forall_{x \in I} \lim_{n \to \infty} f_n(x) = f(x)\]
d.h.
\[\forall_{x \in I} \enspace \forall_{\veps > 0} \enspace \exists_{N(\veps,x)} \enspace \forall_{n>N} \colon \enspace \abs{f_n(x) - f(x)} < \veps\]
}

\dfn{}{
\(f_n \overset{n \to \infty}{\rightrightarrows} f\) gleichmäßig bzgl. \(x \in I\) genau dann, wenn
\[\forall_{\varepsilon > 0} \enspace \exists_{N(\veps)} \enspace \forall_{x \in I} \enspace \forall_{n \geqslant N(\veps)} \colon \enspace \abs{f_n(x) - f(x)} < \veps\]
}

\thm{}{
\(f_n \colon I \to \RR, \enspace n \in \NN, \enspace x_0 \in I\)
\begin{enumerate}[label=(\arabic*)]
	\item \(\displaystyle \forall_{n \in \NN} \lim_{x \to x_0} f_n(x) = \vphi_n\)
	\item \(f_n \overset{n \to \infty}{\rightrightarrows} f, \enspace f \colon I \to \RR\)
\end{enumerate}
\[\implies \quad \exists \lim_{x \to x_0} f(x) = \lim_{n \to \infty} \vphi_n\]
}

\section{Die Folge der Ableitungen}

\thm{}{
\(f_n \in C^1\Le([a,b], \RR\Ri), \enspace \vphi \colon [a,b] \to \RR\)\\
\(f_n \overset{n \to \infty}{\longrightarrow} f\) punktweise für alle \(x \in [a,b]\)\\
\(f_n' \overset{n \to \infty}{\rightrightarrows} \vphi\) gleichmäßig bzgl. \(x \in [a,b]\)\\
\(\implies \quad f \in C^1\Le([a,b], \RR\Ri) \text{ und } \vphi = f'\)
\[\lim_{n \to \infty} \odv{f_n}{x} = \odv{}{x} \lim_{n \to \infty} f_n\]
}

\section{Funktionenreihen}

\(f_n \colon I \sbst \RR \to \RR\)\\
\(S_n(x) = \sum_{k=1}^{n} f_k(x), \enspace x \in I\)

\dfn{}{
Die Reihe \(\sum_{k=1}^{\infty} f_k\) konvergiert punktweise für alle \(x \in I\) genau dann, wenn
\[S_n(x) \overset{n \to \infty}{\longrightarrow} S(x) \enspace \eqcolon \enspace \sum_{k=1}^{\infty} f_k(x)\]
punktweise konvergiert.
}

\dfn{}{
Die Reihe \(\sum_{k=1}^{\infty} f_k\) konvergiert gleichmäßig bzgl. \(x \in I\) genau dann, wenn
\[S_n(x) \overset{n \to \infty}{\rightrightarrows} S(x) \enspace = \enspace \sum_{k=1}^{\infty} f_k(x) \, .\]
}

\thm{}{
\(f_n \in C\Le([a,b], \RR\Ri), \enspace n \in \NN \text{ und } S(x) = \sum_{k=1}^{\infty} f_k(x)\) konvergiere gleichmäßig.
\[\text{Dann ist } S \in C\Le([a,b], \RR\Ri)\]
}

\section{Potenzreihen}

\section{\texorpdfstring{Der Fixpunktsatz von \textsc{Banach}}{Der Fixpunktsatz von Banach}}

\((M,d)\) metrischer Raum, \(M \neq \emptyset\)

\dfn{}{
Man nennt \(T \colon M \to M\) eine Kontraktion, wenn ein \(\alpha < 1\) existiert, sodass für alle \(x, y \in M\)
\[d(Tx, Ty) \leqslant \alpha \cdot d(x,y) \, .\]
}

\thm{}{
Sei \((M,d)\) vollständig und \(T \colon M \to M\) eine Kontraktion. Dann gibt es genau ein \(\tilde{x} \in M\), sodass
\[\underbrace{\tilde{x} = T\tilde{x}}_{\text{Fixpunkt}} \,.\]
}

\end{document}}